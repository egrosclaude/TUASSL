%----------------------------------------------------------------------
%----------------------------------------------------------------------
%----------------------------------------------------------------------
%----------------------------------------------------------------------
%----------------------------------------------------------------------
%			5	PRODUCCION DE SOFTWARE LIBRE
%----------------------------------------------------------------------
%----------------------------------------------------------------------
%----------------------------------------------------------------------
%----------------------------------------------------------------------
%----------------------------------------------------------------------

\section{Producción de y con Software Libre}

\subsection {Adopción estratégica de Software Libre}
\begin{enumerate}
	\item Usar SL
	\item Reportar bugs o solicitudes de características
	\item Aportar patches o código
	\item Convertirse en miembro de la comunidad de usuarios y desarrolladores
\end{enumerate}

\subsection {Desarrollo de proyectos de Software}
\begin{itemize}
	\item Metodologías de análisis y diseño
	\item Etapas básicas
	\begin{itemize}
	\item Estudio de viabilidad
	\subitem En esta fase se considera si el proyecto se
puede realizar, teniendo en cuenta las circunstancias internas y
externas, las diferentes soluciones posibles y los recursos de los
cuales se dispone.
	\end{itemize}
	\begin{itemize}
	\item Análisis
	\subitem Se analizan las necesidades que se desea satisfacer con
el nuevo proyecto, se ajustan los objetivos finales y se centra la solución
tecnológica. También en esta fase se definen las interfaces
entre los diferentes subsistemas que formarán el proyecto y las de
usuario que permitirán interactuar con el sistema.
\end{itemize}
	\begin{itemize}
	\item Diseño
	\subitem En esta fase se realiza el diseño tecnológico de la solución
escogida, proponiendo una arquitectura global y analizando
y estudiando todos los casos de usos (o los más representativos)
existentes.
\end{itemize}
\begin{itemize}
	\item  Desarrollo
	\subitem En esta fase se construye la solución propuesta teniendo
en cuenta el entorno utilizado, se escogen las licencias, se
genera la documentación y se ejecutan las pruebas acordes al
tipo de proyecto y metodología utilizada.
\end{itemize}
	\begin{itemize}
	\item Implantación
	\subitem Se traspasa del entorno de desarrollo al sistema
real y se realizan todas las pruebas y medidas de niveles de prestaciones que conducirán a la aceptación definitiva de proyecto. Se define también el plan de mantenimiento y se toman las decisiones adecuadas para el correcto funcionamiento del sistema durante el resto de su vida.
	\end{itemize}
\end{itemize}


\subsection {Proyecto de software tradicional vs. SL}
\begin{itemize}
\item Modos de desarrollo
	\begin{itemize}
	\item Interno
		\begin{itemize}
		\item Binarios -ocasionalmente fuentes- liberados una vez que se ha validado internamente el desarrollo
		\item Soporte via foros propios o sistema web de tickets
		\item Modificaciones disponibles luego de testing interno, altas demoras
		\item Bases de datos de bugs disponibles solamente para la organización
		\item Desarrollo, validación, soporte, llevados a cabo por el proveedor
		\end{itemize}
	\item Abierto
		\begin{itemize}
		\item Administración de fuentes a través de repositorios públicos (CVS, SVN, GIT)
		\item Soporte via foros, listas, IRC, etc.
		\item Modificaciones disponibles en las listas y repositorios
		\item Páginas de proyecto como documentación básica para los usuarios
		\item Implica relacionarse con nuevas tecnologías y modelos de desarrollo
		\end{itemize}
	\end{itemize}
\item Modos de valuación
	\begin{itemize}
	\item Precio del software tradicional en función de un costo en horas de desarrollo
	\item El desarrollo del SL es soportado por la comunidad
		\begin{itemize}
		\item Luego desaparece la preocupación por el costo de desarrollo
		\item Desplazar el interés económico de la programación a los servicios
		\item Caso extremo: productos ERP
		\end{itemize}
	\end{itemize}
\end{itemize}

\subsection {Modelo de desarrollo Open Source}
\begin{itemize}
	\item El desarrollo está distribuido entre múltiples equipos
	\item Trabajan en diferentes ubicaciones
	\item Estructura flexible, resistente a llegadas o partidas de desarrolladores
	\item Procesos definidos para incorporar e integrar código
	\item Comunicación autodocumentada
	\item Desarrollo asincrónico
	\item Características del producto son desarrolladas incrementalmente
\end{itemize}


	
\subsection {Modelos de negocio}
\begin{itemize}
	\item Roles alternativos
	\begin{itemize}
		\item Desarrollo de SL
		\item Capacitación, instalación, soporte, personalización, consultoría, apoyo 
		\item Integración de sistemas
		\item Distribución de SL (RedHat, Novell)
		\item Distribución de accesorios (libros, CDs)	
	\end{itemize}

	\item Monetización del desarrollo o servicios de distribución
	\begin{itemize}
	\item Publicidad on line
	\item Partnership con empresas de Internet
	\item \emph{Crowdfunding}, \emph{perking}
	\end{itemize}

	\item \url{http://www.bilib.es/recursos/documentacion/}
\end{itemize}


\subsection {Motivaciones para producir SL}
\begin{itemize}
	\item Partir de una solución parcial preexistente
	\subitem Facilitar la puesta en marcha del proyecto
	\subitem Ahorro en costos de análisis y desarrollo
	\subitem Aprovechar la plataforma de usuarios y desarrolladores
	\subitem Acceder a testing masivo y gratuito
	\item Influir en la agenda de desarrollo del proyecto
	\subitem Upstreaming
	\subitem Roles de mantenedor, traductor, documentador, etc.
	\subitem Darle persistencia o diferentes alcances al proyecto
	\item Compartir el esfuerzo de desarrollo con pares
	\subitem Empresas del mismo ramo, organizaciones del mismo nivel
	\subitem Capitalizar las ideas, descubrimientos y desarrollos de los demás
\end{itemize}

\subsection {SL en las organizaciones}
\subsubsection {Administración pública}
\begin{itemize}
	\item La AP administra registros de información que son propiedad de los ciudadanos
	\item Tiene la responsabilidad de custodiarlos y garantizar su privacidad, integridad, persistencia y accesibilidad
	\item Debe seleccionar las mejores herramientas tecnológicas para implementar la política de información, incluyendo aplicaciones, protocolos y formatos de datos
	\item Debe ser cuidadoso en sus inversiones en herramientas tecnológicas
	\item Debe aprovechar los nichos de funcionalidad común en las administraciones locales o regionales, compatibilizando y compartiendo una base de software e información  
	\item \emph{La Administración Pública como receptora de proyectos internos de software libre}, Francesc Rambla i Marigot, 2009, UOC \footnote{\url{http://cv.uoc.edu/autors/MostraPDFMaterialAction.do?id=154680}}
	\item \emph{Razones por las que el Estado debe usar Software Libre}, Federico Heinz, 2001 \footnote{\url{http://proposicion.org.ar/doc/razones.html}}
\nota{	\item Proyecto Munix, Rosario \footnote{\url{http://www.rosario.gov.ar/sitio/gobierno/munix1.jsp}}
	\subitem Objetivos
	\subsubitem Promover el acceso a la información pública a todos los ciudadanos.
	\subsubitem Incrementar el nivel de seguridad en la información.
	\subsubitem Centralizar la administración de la información.
	\subsubitem Revertir la obsolescencia del equipamiento.
	\subsubitem Prevenir posibles irregularidades en el uso de software licenciado.
	\subsubitem Fomentar el desarrollo de la industria de software local.
	}
\end{itemize}
 
\nota {\subsubsection {Educación pública}
}

\subsubsection {Sector privado}
\begin{itemize}
	\item Objetivos
		\begin{itemize}
		\item Competir con líderes de mercado en posición ventajosa
		\item Aumentar la participación en un mercado
		\item Investigar la aceptación del mercado hacia un producto, reduciendo riesgos
		\item Aprovechar promoción gratuita generada por la libre distribución 
		\item Desarrollar un sistema de soporte a bajo costo
		\item Acceder a desarrolladores capacitados
		\item Reducir el tiempo de llegada a los mercados
		\end{itemize}
	\item Modelos de Negocio
	\begin{itemize}

		\item Doble Licenciamiento
		\subitem Este modelo se basa en la distribución de un producto bajo dos licencias distintas: una licencia propietaria tradicional, y una licencia libre restrictiva (tipo GPL). De esta manera, si alguien quiere generar un trabajo derivado, y redistribuirlo sin el código, puede hacerlo, pero deberá pagar una licencia. De lo contrario, todos los trabajos derivados deben redistribuirse con el código.


		\item \emph{Open Core} o núcleo abierto
		\subitem En este modelo existen dos versiones distintas de un programa, una versión básica libre, y una versión comercial propietaria, basada en la anterior, pero con funcionalidad adicional
implementada a través de plugins o accesorios. La versión libre debe emplear
una licencia de tipo MPL o BSD, que permita la combinación para crear un
producto cerrado.

		\item Software como Servicio (SaS)
		\subitem Las empresas desarrolladoras de un producto también podrán explotarlo mediante el paradigma de software como servicio. En lugar de ofrecer servicios de instalación y soporte, la empresa se hace cargo de toda la infraestructura de hardware y software, ofreciendo directamente la funcionalidad a través de Internet. Los ingresos generados, de naturaleza recurrente, toman la forma de subscripciones al servicio.
 
		\item Capacitación y servicios conexos
		\subitem Capitalizar la experiencia de los líderes de proyecto ofreciendo servicios conexos a los productos libres.
	\end{itemize}	
\end{itemize}


