\documentclass[11pt,a4paper]{article}
\usepackage[utf8x]{inputenc}
\usepackage[T1]{fontenc}
\usepackage[spanish]{babel}
\usepackage{amsmath}
\usepackage{amssymb,amsfonts,textcomp}
\usepackage{color}
\usepackage{array}
\usepackage{multirow}
\usepackage{hhline}
\usepackage{hyperref}\hypersetup{pdftex, colorlinks=true, linkcolor=blue, citecolor=blue, filecolor=blue, urlcolor=blue, pdftitle=, pdfauthor=, pdfsubject=, pdfkeywords=}
\usepackage{float}
\usepackage{xkeyval}
\usepackage[pdftex]{graphicx}
\usepackage{appendix}
%\addto\captionsspanish {%
%	\def\appendixname{Apéndices}
%}
% Outline numbering
\setcounter{secnumdepth}{1}
% Reset section numbering between parts
\makeatletter
\@addtoreset{section}{part}
\makeatother  
% List styles
\newcommand\liststyleLi{%
\renewcommand\labelitemi{{\blacksquare}}
\renewcommand\labelitemii{${\circ}$}
\renewcommand\labelitemiii{${\blacksquare}$}
\renewcommand\labelitemiv{{\textbullet}}
}
\newcommand\liststyleLii{%
\renewcommand\labelitemi{{\textbullet}}
\renewcommand\labelitemii{${\circ}$}
\renewcommand\labelitemiii{${\blacksquare}$}
\renewcommand\labelitemiv{{\textbullet}}
}
\newcommand\liststyleLiii{%
\renewcommand\labelitemi{{\textbullet}}
\renewcommand\labelitemii{${\circ}$}
\renewcommand\labelitemiii{${\blacksquare}$}
\renewcommand\labelitemiv{{\textbullet}}
}


% Page layout (geometry)
\setlength\voffset{-1in}
\setlength\hoffset{-1in}
\setlength\topmargin{2cm}
\setlength\oddsidemargin{2cm}
\setlength\textheight{23.246668cm}
\setlength\textwidth{17.006cm}
\setlength\footskip{26.144882pt}
\setlength\headheight{1.016cm}
\setlength\headsep{0.508cm}
% Footnote rule
\setlength{\skip\footins}{0.119cm}
\renewcommand\footnoterule{\vspace*{-0.018cm}\setlength\leftskip{0pt}\setlength\rightskip{0pt plus 1fil}\noindent\textcolor{black}{\rule{0.25\columnwidth}{0.018cm}}\vspace*{0.101cm}}
% Pages styles
\makeatletter
\newcommand\ps@Standard{
  \renewcommand\@oddhead{{\raggedleft Cabecera \ } {\raggedright \thepage{}}}
  \renewcommand\@evenhead{\@oddhead}
  \renewcommand\@oddfoot{}
  \renewcommand\@evenfoot{\@oddfoot}
  \renewcommand\thepage{\arabic{page}}
}
\makeatother
% \pagestyle{Standard}
\usepackage{fancyhdr}
\usepackage{sans}
\pagestyle{fancy}
% footnotes configuration
\makeatletter
\renewcommand\thefootnote{\arabic{footnote}}
\makeatother
\title{Software Libre}
\author{Eduardo Grosclaude}
\date{2013-07-01}
\usepackage{graphicx}

\usepackage{xkeyval}

\usepackage{xcolor}
\newcommand{\revisar}[1]{{\color{red}[#1]}}
%\newcommand{\nota}[1]{{\color{red}[#1]}}
%\newcommand{\revisar}[1]{}
\newcommand{\nota}[1]{}

\newcommand{\quotes}[1]{``#1''}

   
\newcommand{\shade}[1]{\textcolor{black!50}{#1}}

% ancho opcional, por defecto 15cm
% \figura{copyleft}{Símbolo de Copyleft}{copyleft.png}
% \figura[6]{copyleft}{Símbolo de Copyleft}{copyleft.png}
\newcommand{\figura}[4][15]{
 \begin{figure}[htbp] 
 \centering 
 \includegraphics[width=#1cm]{#4} 
 \caption{#3} 
 \label{fig:#2} 
 \end{figure} 
}

% --------------------------------------------------------------------
\begin{document}


\maketitle



\abstract { En este escrito se presenta la descripción y material inicial de la asignatura \textbf{Software Libre}, para la carrera de Tecnicatura Universitaria en Administración de Sistemas y Software Libre, de la Universidad Nacional del Comahue. 

La materia es cuatrimestral en modalidad presencial y las clases son de carácter teórico-práctico, desarrolladas en forma colaborativa. Está preparada con los objetivos generales de \textbf{conocer los aspectos técnicos, legales, económicos y sociales que distinguen al Software Libre y de Código Abierto; conocer las formas de analizar, evaluar y utilizar las fuentes de documentación y soporte del Software Libre y de Código Abierto}. 
 

\newpage
\emph{Página en blanco}
\newpage

\tableofcontents

\newpage
\emph{Página en blanco}

%----------- P R E S E N T A C I O N  ---------
\newpage
\part {La asignatura}


\section{Objetivos}
\subsection{De la carrera}
Según el documento fundamental de la Tecnicatura, el Técnico Superior en Administración de Sistemas y Software Libre estará capacitado para:
\begin{itemize}
	\item Desarrollar actividades de administración de infraestructura. Comprendiendo la administración de sistemas, redes y los distintos componentes que forman la
infraestructura de tecnología de una institución, ya sea pública o privada.
	\item Aportar criterios básicos para la toma de decisiones relativas a la adopción de nuevas tecnologías libres.
	\item Desempeñarse como soporte técnico, solucionando problemas afines por medio de la comunicación con comunidades de Software Libre, empresas y desarrolladores de
software.
	\item Realizar tareas de trabajo en modo colaborativo, intrínseco al uso de tecnologías libres.
	\item Comprender y adoptar el estado del arte local, nacional y regional en lo referente a implementación de tecnologías libres. Tanto en los aspectos técnicos como legales.
\end{itemize}
\subsection{De la asignatura}
\begin{itemize}
\item Conocer los aspectos técnicos, legales, económicos y sociales que distinguen al Software Libre y de Código Abierto
\item Conocer las formas de analizar, evaluar y utilizar las fuentes de documentación y soporte del Software Libre y de Código Abierto
\end{itemize}


\section{Cursado}
\begin{itemize}
	\item Cuatrimestral de 16 semanas, 64 horas totales
	\item Clases teórico-prácticas presenciales
	\item Promocionable con trabajos prácticos
\end{itemize}


\section {Contenidos}
\subsection{Contenidos mínimos}
\begin{itemize}
\item Las licencias de software. Software Libre y Open Source. Comparación. 
\item Ventajas de la disponibilidad del código fuente. 
\item Modelos de desarrollo abiertos y colaborativos. 
\item Aspectos legales y de explotación del Software Libre. 
\item Implantación de sistemas de Software Libre. Factibilidad. 
\item Aspectos económicos y modelos de negocio del Software Libre. 
\item Costo total de operación. Comparación con otras alternativas. 
\item El Software Libre en el sector público, en la educación y en la empresa.
\end{itemize}

\subsection {Programa}
\begin{enumerate}
	\item Introducción. Software Libre y Código Abierto. Aspectos éticos. Implicancias sociales. Localización. Proyectos libres.
	\item Aspectos técnicos. Proyectos de Software Libre. Modelo de desarrollo. Infraestructura tecnológica. Manejo de documentación y soporte. Seguridad. 
	\item Aspectos legales. Dominio Público, Copyright, Copyleft y Licenciamiento. Licencias de FSF, Creative Commons, OSI, otras. 
	\item Uso y aplicación de Software Libre. Costo total de operación. Estudio de costo/beneficio. Procesos de migración. 
	\item Producción de Software Libre y con Software Libre. Modelos de negocio. Colaboración en proyectos. Organizaciones y software.  Administración pública, educación pública, sector privado.
\end{enumerate}

\subsection {Bibliografía inicial}
\begin{itemize}
	\item \textbf{Introducción al Software Libre}, Jesús González Barahona, Joaquín Seoane Pascual y Gregorio Robles
	\item \textbf{Aspectos legales y de explotación del software libre}, Malcom Bain, Manuel Gallego Rodríguez, Manuel Martínez Ribas y Judit Rius Sanjuán
	\item \textbf{Guía práctica sobre Software Libre, su selección y aplicación local en América Latina y el Caribe}, Fernando Da Rosa y Federico Heinz
\end{itemize}



\section{Evaluación}
La evaluación de la materia se realizará mediante trabajos grupales de investigación y desarrollo sobre proyectos de Software Libre, de la siguiente manera.
\begin{itemize}
	\item Los estudiantes se dividirán en grupos de 2 a 5 personas. 
	\item Los grupos desarrollarán trabajos prácticos en etapas que se distribuirán a lo largo de la materia. 
	\item Cada grupo abrirá un diario, blog o wiki de acceso público en cualquier sitio disponible y publicará, mediante el Foro de la materia, la forma de acceder al diario para lectura. Los docentes y los demás estudiantes de la materia podrán acceder al diario del grupo para lectura. Todo cambio en la dirección o forma de acceso deberá ser informado mediante el Foro.
	\item El grupo irá aportando los resultados de cada etapa de los trabajos a su diario, y periódicamente comentará además en clase las experiencias surgidas durante la realización de los trabajos.
	\item El material publicado en el diario será reunido en un documento final que será entregado \textbf{en formato electrónico} al finalizar la materia. El documento indicará tema del trabajo, resumen, integrantes del grupo, desarrollo y conclusiones. 
	\item El documento será acompañado por una presentación de no más de treinta minutos que será expuesta según el cronograma adjunto. 
	\item La acreditación final tendrá en cuenta la calidad del material aportado al diario por el grupo, la calidad de los documentos finales de los trabajos, la presentación oral y la participación en clase ofreciendo la experiencia adquirida durante la realización de los trabajos.
\end{itemize}

\subsection {Trabajo I - Colaboración con proyectos libres}
\subsubsection{Etapa 1}  
Descargar e instalar software ofrecido por un proyecto de Software Libre que esté en actividad (puede tratarse de un entorno de escritorio, un programa de sistema, programas de usuario final, una distribución completa, etc.). Familiarizarse con el software utilizándolo. 
\subsubsection{Etapa 2} 
Basándose en el conocimiento adquirido con el uso del software, colaborar de alguna forma con el proyecto que lo origina: 
\begin{itemize}
	\item traduciendo o localizando parte del software,
	\item generando documentación faltante, 
	\item traduciendo parte de la documentación, 
	\item detectando y denunciando errores en el software o en la documentación,
	\item aportando, modificando o corrigiendo código,
	\item aportando conocimiento a los usuarios del proyecto en blogs, salas de chat, bases de conocimiento, etc.
\end{itemize}
Puede abordarse cualquier cantidad manejable de proyectos. La colaboración debe consistir en alguna interacción positiva y completa con cada proyecto. El grupo incorporará al diario los reportes que acrediten esa interacción. Cuando no sea posible realizar o completar la interacción se indicarán las causas, y las acciones realizadas.

El aporte al proyecto debe efectuarse por los canales establecidos por el proyecto. Si se trata de documentación, respetar el formato utilizado; si es el reporte de un error, hacerlo por la vía preferida por el proyecto, etc.

\subsubsection{Etapa 3} 
El grupo entregará un documento conteniendo la historia de las interacciones con cada proyecto, adjuntando las pruebas en anexos y ofrecerá una presentación.

\subsection {Trabajo II - Evaluación de proyectos libres}

\subsubsection{Etapa 1} 
El grupo enunciará un determinado requerimiento concreto de software que puede ser presentado por un empleador. Algunos ejemplos posibles son:
\begin{itemize}
	\item \quotes{un servidor de correo electrónico que maneje listas},
	\item  \quotes{una aplicación de control de asistencia para empleados},
	\item  \quotes{un sistema de edición de textos para traductores},
	\item  \quotes{un sistema de gestión de contenidos web que incluya workflow}, 
	\item \quotes{un motor de juegos 2D para crear juegos que asistan en la enseñanza de matemática},
	\item  \quotes{un programa de simulación de ataques para evaluar postura de seguridad}, 
	\item \quotes{un sistema de control de stock para zapaterías},
	\item \quotes{una distribución de GNU/Linux para escuelas de arte},
	\item \quotes{una distribución para sistemas empotrados}, etc.
\end{itemize}
El grupo debe comprender el propósito del software requerido y debe contar con al menos un integrante con conocimiento razonable de la temática involucrada. El grupo escribirá una entrada en el diario consignando toda la información posible sobre los requerimientos. 

\subsubsection{Etapa 2} 

\begin{itemize}
	\item El grupo $n$ (en adelante \quotes{el proveedor}) tomará a su cargo el requerimiento del grupo $n+1$ (en adelante \quotes{el cliente}), y se atendrá a dicha descripción para el resto del trabajo. 
	\item El grupo proveedor buscará proyectos de SL que apunten a cubrir esos requerimientos y seleccionará al menos dos proyectos, idealmente tres, de entre ellos.
\end{itemize}

\subsubsection{Etapa 3}
Los proyectos serán comparados en función de varios parámetros o dimensiones.
\begin{itemize}
	\item  ajuste a los requerimientos (actual, previsto o potencial),
	\item  licenciamiento, 
	\item  motivación del desarrollo, 
	\item  modelos de negocio del proyecto, 
	\item  tamaño y permanencia de la comunidad,
	\item  dinámica de soporte, 
	\item  dinámica de actualizaciones y mejoras del software.
\end{itemize}

Se podrán agregar a la comparación uno o más desarrollos no libres. 

Las dudas sobre detalles de los requerimientos serán dirigidas al grupo cliente, y contestadas por aquél, mediante el Foro de la página de la materia.  
\subsubsection{Etapa 4} 
El grupo entregará un documento conteniendo la comparación y haciendo una recomendación final, explicando sus fundamentos. Deberán volcar en el trabajo lo que se vaya aprendiendo durante el curso de la materia, en cada uno de los parámetros o dimensiones nombrados. Finalmente ofrecerán una presentación sobre el trabajo.

\label{sub:acreditacion}

\subsection {Cronograma de ejecución}
\begin{tabular}{c|l|l|l}
Semana & Unidad & Trabajo I & Trabajo II\\
\hline
\hline
1	& 	1. Introducción, Software Libre & Etapa 1 &  \\
2 	& 								 	& \\
\hline
\hline
3	& 	2. Aspectos técnicos			& Etapa 2 &  \\
4 	& 									&\\
5	& 									&\\
6	& 									&\\
\hline
\hline
7 	& 	3. Aspectos legales				& Etapa 3 \\
8	& 									& Entrega y presentaciones\\ 
9	& 									& & Etapas 1 y 2\\
\hline
\hline
10	& 	4. Uso de SL						&& Etapa 3\\ 
11	& 									& \\
12	& 									&\\
13	& 									&\\
\hline
\hline
14	& 	5. Producción de SL				&& Etapa 4\\
15	& 									&\\
16	& 									&& Entrega y presentaciones\\ 
\hline
\end{tabular}



% \begin{tabular}{|r|c|c|c|c|c|c|c|c|}
% \hline
%\textsf{7} & fbox {algo} & & & & & & &\\ 
%\hline
%\textsf{7} & & & & & & & &\\ 
%\hline
%\end{tabular}

% subsection  (end)


%----------- M A T E R I A L ---------
\newpage
\part {El Software Libre}

%--------------------------------------------------------------------
%			1 ASPECTOS ETICOS Y SOCIALES
%--------------------------------------------------------------------

\nota {
	Introducción. Software Libre y Código Abierto. Aspectos éticos. Implicancias sociales. Proyectos libres. Localización.
}
\section{Software Libre y Código Abierto}



\nota {Historia anterior}
\nota {Ética - Valores - Licencias - Libertades - Código Fuente -
Contenidos libres - Programas ejecutables - Distribuciones -
Estándares abiertos - Desarrolladores - Usuarios - Soporte - Organizaciones }

\begin{itemize}
	\item Software Libre, Open Source/Código Abierto, FOSS o FLOSS
	\subitem \url{http://drupal.usla.org.ar/page/%C2%BFque-es-el-software-libre}
	\subitem Free $\neq$ Gratis
	\item El proyecto GNU y la FSF
	\subitem \url{http://www.fsfla.org}
	\subitem \url {http://es.wikipedia.org/wiki/Portal:Software_libre}
	\item Las cuatro libertades
\begin{enumerate} \setcounter{enumi}{-1}
	\item De correr el software, con cualquier propósito
	\item De estudiar cómo está hecho, para poder adaptarlo a sus propias necesidades
	\item De copiarlo y darlo a otras personas para poder ayudarlas
	\item De mejorarlo y donar el resultado a la comunidad, para permitir el avance colectivo
\end{enumerate}
	\item Código fuente, código objeto
	\item Open Source Initiative
	\subitem \url{http://opensource.org/} 
	\item The Open Source Definition - \url{http://www.opensource.org/docs/osd}
	\begin{enumerate}	\item Libre Redistribución
	\item Código Fuente
	\item Obras derivadas
	\item Integridad de los fuentes del autor
	\item No Discriminación contra personas o grupos
	\item No Discriminación contra campos de actividad
	\item Distribución de la licencia
	\item La licencia no debe ser específica de un producto
	\item La licencia no debe restringir el uso de otro software
	\item La licencia debe ser tecnológicamente neutra
\end{enumerate}
\end{itemize}


\figura {mapa}{Mapa conceptual del Software Libre (René Mérou)}{Mapa_conceptual_del_software_libre.png}



\subsection {Aspectos éticos}
\begin{itemize}
	\item Valores
	\subitem Cooperación más importante que individualidades
	\subsubitem \quotes{Si los dos tenemos una manzana/una idea}
	\subsubitem \quotes{Si no está hecho es porque no lo hiciste}
	\subsubitem \quotes{Con suficientes ojos se detectan todos los errores}
	\subitem Puedo aprovechar lo que me ofrecen libremente en
lugar de \emph{piratear}
	\subitem Puedo comprender los mecanismos, trascender la herramienta y acceder a los conceptos
	\subitem Puedo ser parte del desarrollo de la tecnología y cooperativamente contribuir al mejoramiento

	\item Implicancias sociales
	\subitem Equilibrar la balanza de pagos internacional del conocimiento
	\subitem Crítica al concepto de Propiedad Intelectual
		\subsubitem \url{http://www.vialibre.org.ar/}
	\subitem Traducciones y localizaciones
	\subitem Uso correcto de los dineros públicos
	\subitem Transparencia de uso de la información pública
	\subitem Gobierno electrónico
	\subitem Hacktivismo
	\subsubitem Uso de computadoras para promover fines políticos, principalmente libertad de expresión, derechos humanos y ética de la información
	\subsubitem Empoderamiento de las minorías
	\subsubitem Wikileaks
	\subsubitem Diaspora
\end{itemize}


\subsection{Proyectos Libres}

Proyectos que generan conocimiento libre

Cualquier proyecto, sobre cualquier temática, ligado a licencias que permitan el uso, copia, modificación y distribución libre de los conocimientos o la información que allí confluyen. Unen a personas con iguales objetivos o problemáticas, que comparten trabajo y hacen públicos y libres sus resultados.

Libertades
\begin{itemize}
	\item de usar el conocimiento generado en el proyecto para cualquier fin
	\item de estudiar el proyecto y adaptarlo a las propias necesidades
usando la información generada por el proyecto
	\item de redistribuir copias de esa información, de manera que otros se beneficien de ella
	\item de mejorar el proyecto y hacer públicas las mejoras a los demás, de modo que toda la comunidad se beneficie de ellas
	\subitem Open Source Hardware
	\subitem Redes Libres
	\subitem Project Gutenberg
	\subitem Wikipedia, Wikimedia
	\subitem Libros libres, música libre
\end{itemize}



\subsection{Preguntas}
\begin{enumerate}
	\item ¿Cómo se autodefine la Free Software Foundation (FSF)?
	\item ¿En qué consisten las cuatro libertades definidas por FSF?
	\item ¿Puede imaginar una definición diferente de libertades? ¿Puede enunciar otro conjunto de libertades que garanticen los derechos del usuario considerados importantes por FSF?
	\item ¿Cómo se resumen las diferencias entre código fuente y código binario, objeto o ejecutable? ¿Todo código ejecutable es no fuente? ¿Todo archivo fuente es de código?
	\item ¿Por qué es importante la libertad de acceso al código fuente?
	\item ¿Qué significa \emph{privativo} en la terminología de FSF?
	\item ¿De qué manera el software propietario o privativo vulnera las libertades establecidas por FSF?
	\item ¿De qué manera las licencias libres impiden que las empresas que producen software propietario puedan apropiarse del trabajo de quienes desarrollan proyectos libres?
	\item ¿Qué es Copyright? ¿En qué momento aparece el Copyright y a quién pertenece? 
	\item ¿Qué es Copyleft? ¿Cuál es la diferencia entre Copyright y Copyleft?
	\item ¿Qué es \emph{dominio público}? 
	\item El software Windows 8 es descargable de un sitio web\footnote{\url{http://windows.microsoft.com/en-au/windows/download-shop}}. Verdadero o falso:  
	\begin{enumerate} 
		\item Luego, es software libre.
		\item Debe pagarse para poder usarlo, luego no es software libre. 
	\end{enumerate}

	\item ¿Son ejemplos de Software Libre los siguientes?
\begin{itemize}
	\item Internet Explorer, Firefox, Chrome, Opera
	\item Microsoft Office, Libre Office
	\item Adobe Reader, evince, atril
	\item Un ERP como Adempiere o SAP
	\item Tango Gestión, de Axoft
	\item MySQL, Microsoft SQL Server
	\item El web server Apache, el web server IIS
	\item Moodle, Joomla, Plone, Drupal
	\item Android, OSX, ClearOS, Solaris
	\item Aplicaciones que conozca para smartphones 
	\item Un juego web cualquiera que conozca
	\item Extensiones de Chrome para aprender idiomas u otros utilitarios 
	\item Drivers para impresoras
	\item El software web de administración de un router
\end{itemize} 



\end{enumerate}


%----------------------------------------------------------------------
%----------------------------------------------------------------------
%			2 ASPECTOS TECNICOS
%----------------------------------------------------------------------
%----------------------------------------------------------------------
\nota {	\item Aspectos técnicos. Proyectos de Software Libre. Modelo de desarrollo. Infraestructura tecnológica. Manejo de documentación y soporte. Seguridad.
}
 
\section{Aspectos técnicos}
\subsection {Código fuente y ejecutables} 
\begin{itemize}
	\item Lenguajes de programación
	\subitem Lenguajes compilados
	\subitem Lenguajes interpretados
	\item Formatos de archivos y documentos
	\subitem Archivos legibles por humanos
	\subsubitem Documentación: READMEs, Markup Languages, .pod, TeX/LaTeX, .rst...
	\subsubitem Configuración o datos: XML, YAML, .cfg,...
	\subitem Archivos estructurados
	\subsubitem Formatos abiertos y propietarios
	\subsubitem Formato .DOC, formato .odt
	\item Protocolos e interfaces
	\subitem Protocolos de Internet vs. protocolos propietarios
	\subitem RFCs 
	\item Bibliotecas
	\subitem Linking o vinculación
	\subitem Linking dinámico, Shared Objects (.so), DLLs
\end{itemize}

\subsection {Ciclo de compilación}
\figura[10]{ciclo}{Ciclo de compilación y ejecución}{ciclo_de_compilacion.eps}

\figura[8]{fuente}{Código fuente en lenguaje C}{bibliotecas-src-0.eps}

\figura{objeto}{Código objeto procedente del programa en C}{objeto.png}


\figura[8]{bibsrc1}{Un archivo fuente que usa dos funciones externas}{bibliotecas-src-1.eps}

\figura[8]{bibsrc2}{Archivo fuente conteniendo funciones}{bibliotecas-src-2.eps}

\figura[10]{bib1}{Compilación de los archivos fuente generando módulos objeto}{bibliotecas-1.eps}

\figura[10]{bib2}{Vinculación estática de objetos generando un ejecutable}{bibliotecas-2.eps}

\figura[10]{bib3}{Creación de una biblioteca compartida (\emph{shared object})}{bibliotecas-3.eps}

\figura[8]{bib4}{Uso de bibliotecas de vinculación dinámica}{bibliotecas-4.eps}

\figura[10]{bib5}{Compartiendo funciones de biblioteca}{bibliotecas-5.eps}

\subsection {Proyectos de Software Libre}
\begin{itemize}
	\item Motivación
	\item Perfil de los desarrolladores
	\item Modelo de desarrollo
	\item Comunidad y vitalidad
\end{itemize}

\subsubsection{La Catedral y el Bazar }
\begin{itemize}
	\item La Catedral y el Bazar vs. la Ingeniería de Software
	\item Representa la postura del movimiento Open Source
	\item Historia informal y razonada de un proyecto de SL
	\item Dos modelos de desarrollo
	\item Motivación y equipo de desarrollo inicial 
	\item Reutilizar antes que desarrollar desde cero
	\item Roles de los usuarios
	\item Liberar rápido y con frecuencia
\end{itemize}

\begin{tabular}{c|c|c}
 & Catedral & Bazar  \\
\hline
\hline
Gobierno & Vertical & Democrático \\
\hline
Roles 	 & Asignados & Libres \\
		 & Estáticos & Móviles \\
\hline
Usuarios & Clientes & Co-desarrolladores \\
\hline 
Procesos & Definidos & Multiestratégicos \\
\hline 
Entregas & Planificadas  	& \quotes{Cuando esté listo} \\
		 & Poco frecuentes 	& \emph{Release early}, \\
		 &				 	& \emph{Release often} \\
\hline
\end{tabular}

% --------------------------------------------------------------



\nota {Desarrollo abierto y colaborativo}

\subsection{Infraestructura tecnológica}
\begin{itemize}
	\item Internet y RFCs
	\item Freecode \url{http://freecode.com}
	\item Sourceforge \url{http://sourceforge.net}
	\item Google Code \url{http://code.google.com}
	\item GitHub \url{http://github.com}
\end{itemize}


\figura{freecode}{Nube de tópicos (\emph{tags}) de Freecode.org}{freecode.png}

\figura{github}{Interfaz del repositorio Github}{github.png}

\figura{googlecode}{Interfaz del repositorio Google Code}{googlecode.png}

\figura{centos}{Interfaz de edición del wiki de una distribución GNU/Linux}{centos.png}

\nota{
\subsection {Documentación y soporte}
\nota{CentOS,GLPI,Bandwidth Arbitrator,mosshe,tcng,zim,Libre Office }
\nota{Libro libre de Redes Olivier Bonaventure}
\subsection {Seguridad}
}



\subsection {Preguntas}

\begin{enumerate}
	\item ¿Cómo se pueden resumir las diferencias entre \emph{la catedral} y \emph{el bazar}?
	\item ¿Por qué en \emph{La Catedral y El Bazar} se hace énfasis en la capacidad de reutilizar código? ¿Qué relación especial tiene esta capacidad con el SL? 
	\item ¿Cómo suele tener origen un proyecto de SL? ¿Cuál suele ser el disparador o motivador de la creación de un proyecto nuevo? ¿En qué casos se considera necesario comenzar un proyecto nuevo en lugar de aprovechar proyectos existentes?
	\item ¿Qué roles asigna el modelo del Bazar a los usuarios del software?
	\item ¿A qué se llama un \emph{dictador benevolente}?
	\item ¿Por qué el modelo del Bazar recomienda la regla \quotes{liberar rápido y con frecuencia}?
	\item ¿En qué consiste la fase de captura de requerimientos de un proyecto de software? ¿En qué momento de la vida del proyecto se da esta fase?
	\item ¿A través de qué herramientas capturan requerimientos los proyectos de SL que usted conozca, en especial, los que su grupo está considerando para los trabajos prácticos de la materia?
	\item ¿Qué herramientas de comunicación suelen utilizar los desarrolladores de SL? ¿Qué funciones cumplen estas herramientas (de comunicación interpersonal, de transmisión de conocimiento técnico, de transmisión o análisis de código, de descripción de procesos...)?
	\item ¿En qué consiste la Ley de Brooks y por qué se dice que la complejidad de la interacción entre los desarrolladores se incrementa cuadráticamente? 
	\item Según Raymond, ¿de qué modo Internet y el SL se oponen a la Ley de Brooks? ¿A qué se llama el efecto Delphi?
	\item ¿Qué crítica se hace a esta última afirmación de Raymond, según el material \emph{Introducción al Software Libre} de Barahona y otros? 
	\item ¿Cuáles son los rasgos que hacen a los mejores líderes de proyecto, según Raymond?
\end{enumerate}



%----------------------------------------------------------------------
%----------------------------------------------------------------------
%----------------------------------------------------------------------
%			3 	ASPECTOS LEGALES
%----------------------------------------------------------------------
%----------------------------------------------------------------------
%----------------------------------------------------------------------
\nota {
	\item Aspectos legales. Dominio Público, Copyright, Copyleft y Licenciamiento. Licencias de FSF, Creative Commons, OSI, otras. 
}



\section{Aspectos legales}
\nota {Ética hacker - \url{http://es.wikipedia.org/wiki/Partido_Pirata} - Torrent

\url{http://www.cultofmac.com/221474/hackers-can-grab-control-of-your-camera-via-its-wi-fi-sd-card}
}
\figura{categorias}{Categorías de software según FSF}{categorias.png}

\subsection{Las licencias de software}
\begin{itemize}
	\item Código, código fuente, código objeto, código ejecutable
	\item Documentación fuente y en otros formatos
	\item Dominio Público
	\item El concepto de Copyleft
	\item Las licencias de FSF
	\item The Open Source Definition
	\subitem \url{http://opensource.org/docs/osd}
	\item Creative Commons
	\subitem \url{http://www.creativecommons.org.ar/}
\end{itemize}



\figura[4]{copyleft}{Símbolo de Copyleft}{copyleft.png}
\figura[6]{commons}{Logotipo de Creative Commons}{CC-logo.png}

\subsection{Preguntas}
\begin{enumerate}
	\item El acceso al código fuente, ¿implica automáticamente la posibilidad de introducir modificaciones y mejoras?  ¿Y la de redistribuir el software? 
	\item Si modifico un software con licencia libre, ¿estoy obligado por la licencia libre a redistribuirlo públicamente? 
	\item En caso de redistribuir software liberado bajo licencia libre, ¿a qué me obliga una licencia libre? ¿A qué se obliga la persona que lo recibe de mí?
	\item Si descargo un software publicado bajo una licencia libre, ¿puedo redistribuirlo? ¿Puedo cobrar por redistribuirlo? ¿Y si se trata de una licencia Open Source?
	\item Si descargo un software publicado bajo una licencia libre, ¿puedo modificarlo? ¿Puedo cobrar por modificarlo? ¿Y si se trata de una licencia Open Source?
	\item ¿En qué consisten las licencias de software permisivas y qué ejemplos se pueden dar?
	\item ¿Cuál es la opinión de FSF respecto del movimiento Open Source?
	\item ¿Cómo puede resumirse la diferencia entre los conceptos de Free Software y Open Source?
	\item Mencione proyectos o productos Open Source que no sean considerados libres por FSF.
	\item ¿Qué licencias reconoce y no reconoce FSF?
	\item ¿Cuál ha sido la historia de versiones de GPL hasta el momento y qué motivó cada versión? ¿Considera que puede ser necesaria una nueva versión en algún momento futuro?
	\item ¿A qué se llama en inglés \emph{appliances}? ¿Qué ejemplos conoce? 
	\item ¿Qué es el producto Tivo y de qué manera se relaciona con la historia de la licencia GPL?
	\item ¿Cuándo un dispositivo \emph{appliance} respeta las libertades, según FSF?
	\item ¿Puede identificar desarrollos originalmente libres que fueron absorbidos por intereses corporativos? ¿Qué licencias utilizaron?
	\item ¿Puede mencionar dispositivos \emph{appliance} que respeten y no respeten las libertades según FSF?

	\item ¿Qué significa \emph{jailbreak} y cuál es la posición legal de las empresas al respecto? ¿Qué es 'rooting'? ¿Qué significa 'ingeniería inversa'? ¿Es una actividad legal? 

	\item ¿Qué significa DRM? ¿De qué manera afecta DRM a las libertades del usuario sgún FSF?

\end{enumerate}

%----------------------------------------------------------------------
%----------------------------------------------------------------------
%----------------------------------------------------------------------
%----------------------------------------------------------------------
%			4	USO Y APLICACION
%----------------------------------------------------------------------
%----------------------------------------------------------------------
%----------------------------------------------------------------------
%----------------------------------------------------------------------
\nota {
	\section{Implantación de sistemas de Software Libre}
	\item Uso y aplicación de Software Libre. Costo total de operación. Estudio de costo/beneficio. Procesos de migración. 
}

\section{Uso y aplicación de Software Libre}

\nota{
\subsection{Aspectos económicos}
\begin{itemize}
	\item Productos, Soluciones y Servicios
	\item Modelos de negocio\footnote{\url{http://www.fsfla.org/~lxoliva/papers/free-software/beautiful-mind.pdf}}
 

	\item Proceso de migración
	\item Costo total de operación
\end{itemize}
\subsection{SL en el sector público}
\label{sub:SLenelsectorpúblico}
}

\subsection {Selección de Software Libre}
\begin{itemize}
	\item Cualidades del software
		\subitem Proyecto vivo y en actividad
		\subitem Proyecto más aceptado
		\subitem Estable y maduro
		\subitem Tendencia
	\item Cualidades de la organización
		\subitem Impacto que provocará el cambio
		\subitem Estudio de Costos/Beneficios 
		\subitem ${C_1/B_1 \longleftrightarrow C_2/B_2}$
		\subitem Estudio de Costo Total de Propiedad de ambas soluciones
		\subitem Riesgos
		\subsubitem Taxonomía de riesgos (ver Anexo \ref{sec:CuestionarioRiesgos})
	\item Valuación de costos, beneficios, riesgos
	\subitem Difícil en forma absoluta
	\subitem Cuando se trata de comparar se puede fijar una escala adimensional
	\subitem Los riesgos se pueden computar como ${probabilidad de ocurrencia * gravedad}$
\end{itemize}

\subsection {Administración de TI}
\begin{itemize}
	\item Frameworks de administración de Tecnologías de Información (TI)
	\item Compendios de mejores prácticas
		\subitem CobiT \url{}
			\subsubitem TI en general
			\subsubitem Libremente reproducible
		\subitem ITIL \url{}
			\subsubitem Servicios
			\subsubitem Más detallada
			\subsubitem Licencia restrictiva
		\subitem ISO
\end{itemize}

\subsection {CobiT}
\begin{itemize}
	\item CobiT define 34 procesos de TI. Incluye herramientas para medición de desempeño, una taxonomía de factores críticos de éxito, y modelos de madurez para asistir en la toma de decisiones hacia el mejoramiento de capacidades. 
	\item Objetivos de CobiT
		\subitem Asegurar que 
		\subsubitem TI se alinea con la actividad de la organización
		\subsubitem TI habilita las operaciones y maximiza los beneficios
		\subsubitem Los recursos se usen responsablemente
		\subsubitem Los riesgos se manejen adecuadamente
	\item Dominios definidos en CobiT
		\subitem Plan and Organise (PO)
		\subitem Acquire and Implement (AI)
		\subitem Deliver and Support (DS)
		\subitem Monitor and Evaluate (ME)
	\emph {
		\subitem \quotes{Acquire and Implement, AI}
		\subsubitem Identify Automated Solutions
		\subsubitem Acquire and Maintain Application Software
		\subsubitem Acquire and Maintain Technology Infrastructure
		\subsubitem Enable Operation and Use
		\subsubitem Procure IT Resources
		\subsubitem Manage Changes
		\subsubitem Install and Accredit Solutions and Changes
	}
\end{itemize}

\figura {cobit}{Un ejemplo de guía de actividades de CobIT}{CobIT-RACI.png}





\subsection {Migraciones en general}
\begin{itemize}
	\item Difícil por la naturaleza de los sistemas privativos
	\item Apoyo de los usuarios
	\subitem Se necesitan defensores del cambio
	\subitem Información para involucrarlos
	\item Diagnosticar claramente el escenario de partida
	\subitem Hardware, software, configuración de la red, usuarios
	\item Definir la situación final deseada
	\item Justificar la migración
	\subitem Análisis de costos de la migración y proyección de ahorros factibles
	\item Planificar la migración
	\item Migración de los datos
	\item Preparar y capacitar el equipo técnico
\end{itemize}

\subsection {Formas de migración a SL}
\begin{itemize}
	\item Repentina
		\subitem Riesgosa
		\subitem Requiere gran inversión en capacitación
	\item Progresiva
	\begin{enumerate}
		\item Usando SL bajo el sistema operativo existente
		\item Migrando los datos progresivamente
		\item Migrando los ambientes y sistemas operativos	
	\end{enumerate}
\end{itemize}

\subsection {Facilitar migración futura}
\begin{enumerate}
	\item Usar Formatos y estándares abiertos
	\item Usar metodologías basadas en capas para el desarrollo, separando código de interfaz y de acceso a los datos 
	\item Generar aplicaciones portables, evitando lenguajes de arquitecturas específicas
	\item Evitar la construcción de aplicaciones que requieran otras privativas
	\item Desarrollos web que cumplan estándares W3C y validados contra navegadores libres
\end{enumerate}

\subsection {Costo Total de Propiedad, CTP}
\begin{itemize}
	\item O también \emph{Total Cost of Ownership, TCO}
	\item Cálculo para facilitar la evaluación de costos directos e indirectos asociados con la adopción de un componente de IT 
	\item Evaluar beneficios de la migración frente a los costos
	\item El CTP define el costo total de la propiedad para el uso de una tecnología concreta durante el período de vida de dicha tecnología
	\subitem Medición del impacto
	\item Taxonomías de costos
	\subitem Costos de adquisición
	\subitem Costos asociados con reparaciones
	\subitem Costos de oportunidad
	\subitem Costos extendidos
	\subsubitem Servicios conexos de soporte, redes, seguridad, capacitación, licenciamiento de software u otros componentes
	\item ¿Taxonomías de \emph{beneficios}?
\end{itemize}

\subsection {Metodologías de análisis de CTP}
\begin{itemize}
	\item Una metodología genérica
	\subitem \url{http://blogs.msdn.com/b/eduardop/archive/2006/05/29/610441.aspx}
	\subitem Considera costos divididos en iniciales y operativos
	\subitem Costo inicial de la solución ${CI = CH + CS + CINS + CCON}$
	\subsubitem CH = Costo del hardware
	\subsubitem CS = Costo del software
	\subsubitem CINS = Costo de servicios iniciales de instalación
	\subsubitem CCON = Costo de servicios iniciales de configuración

	\subitem Costo de administración (CA)
	\subsubitem Personal dedicado al mantenimiento operativo
	\subsubitem Cálculos basados en honorarios y gastos por personal

	\subitem Costo de operación (CO)
	\subsubitem Pérdidas por caída de operación o soporte reactivo
	\subsubitem Cálculos basados en costo de incidentes

	\subitem Costo de soporte (CS)
	\subsubitem Costo de proveer asistencia a usuarios
	\subsubitem Basado en costo por hora de soporte
	
	\subitem TV = Tiempo de vida de la solución
	\subitem Costo total ${= CI + (CA + CO + CS) * TV}$


	\item Otra metodología orientada a SL
	\subitem Centro de Excelencia en Software Libre U. Castilla-La Mancha
	\subitem \url {http://www.bilib.es/documentos/Taller_de_Migracion.pdf}
	\subitem Costo Total CT = CD + CI = Costos Directos + Costos Indirectos
	\subsubitem Directos se producen como consecuencia de la adquisición del software
	\subsubitem Indirectos se producen como consecuencia de pérdidas o caídas de productividad
	\subitem ${CD = CH + CS + CSOP + CFOR + CPER}$
	\subsubitem CH = Costo del hardware
	\subsubitem CS = Costo de licenciamiento de software
	\subsubitem CSOP = Costo de soporte = Instalación + Configuración + Mantenimiento
	\subsubitem CFOR = Costo de formación del personal de operación
	\subsubitem CPER = Costo de personal de operación
	\subitem ${CI = CM + CC + CSEG + CE + CDISP}$
	\subsubitem CM = Costo de mantenimiento por errores o problemas del software
	\subsubitem CC = Costo de oportunidad debido a caídas de sistema
	\subsubitem CSEG = Costo de aseguramiento informático
	\subsubitem CE = Costo de asegurar la escalabilidad
	\subsubitem CDISP = Costo de asegurar la disponibilidad

\end{itemize}
%----------------------------------------------------------------------
%----------------------------------------------------------------------
%----------------------------------------------------------------------
%----------------------------------------------------------------------
%----------------------------------------------------------------------
%			5	PRODUCCION DE SOFTWARE LIBRE
%----------------------------------------------------------------------
%----------------------------------------------------------------------
%----------------------------------------------------------------------
%----------------------------------------------------------------------
%----------------------------------------------------------------------

\section{Producción de y con Software Libre}

\subsection {Motivaciones para producir SL}

\subsection {Modelos de negocio}
\nota {Crowdfunding}

\subsection {Colaboración con proyectos libres}

\subsection {SL en las organizaciones}
\subsubsection {Administración pública}
\subsubsection {Educación pública}
\subsubsection {Sector privado}



%----------- A N E X O S ---------
\newpage
\part {Anexos}
\appendix

\section {Licencias libres y no libres}
\subsection{Comparación de GPL y MS EULA}

\subitem{EULA: End User License Agreement
\subitem{GPL: General Public License}
\subitem \url{http://blog.desdelinux.net/eula-windows-vs-gpl-linux-duelo-de-licencias}

\subsubsection{Características básicas de la EULA}
\shade{
Prohibida su copia y redistribución (copyright).
Puede ser usada por una sola computadora con un máximo de dos procesadores.
No puede ser utilizado como servidor web o como servidor de archivos.
Requiere registro después de 30 días.
Podría dejar de funcionar si se realizan cambios de hardware.
Las actualizaciones pueden cambiar la EULA si la compañía así lo decidiera.
Puede ser transferida al nuevo usuario una sola vez. El nuevo usuario debe estar de acuerdo con los términos de uso (EULA).
Impone limitaciones a la reingeniería inversa.
Se conceden permisos a Microsoft para tomar información sobre el Sistema y su uso.
Se conceden permisos a Microsoft para proveer esta información a otras organizaciones.
Se conceden permisos a Microsoft a realizar cambios a el sistema sin el consentimiento del usuario.
Garantía por los primeros 90 días, Actualizaciones, reparaciones y parches no tienen garantía. 
}
\subsubsection {Características básicas de la licencia libre GPL}
\shade {
Libertad de copiar, modificar y redistribuir el software.
Impide que un grupo o ente impida que otro grupo o ente no pueda tener estas mismas libertades.
Provee cobertura a los derechos de los usuarios de copiar, modificar y redistribuir el software.
Se puede vender si el usuario así lo decide, y los servicios conexos a dicho software pueden ser cobrados. 
Toda patente debe ser licenciada para el uso de todos o no ser licenciada en absoluto.
Software modificado no debe llevar costo de licencias.
Se debe proveer con el código fuente.
Si hay un cambio en la licencia, los términos generales de la licencia existente se mantienen. 
}
\section {Política de uso de SL en la UNC}

\subsubsection{Establecen la política de uso de Software Libre en la UNCo}

\emph {Martes, 13 de Diciembre de 2011 12:34}

\emph {Última actualización el Martes, 13 de Diciembre de 2011 20:10 }

 \emph {Escrito por Prensa UNCo}

Mediante la Ordenanza Nº 590 del 13 de diciembre de 2011, el Consejo Superior de la Universidad Nacional del Comahue resolvió establecer como política en el ámbito administrativo de la Casa de Altos Estudios el uso de Software Libre desarrollado con estándares abiertos en todos sus sistemas y equipamientos informáticos.

La medida fue tomada luego de que la Facultad de Informática elevara una solicitud en ese sentido.

 


En el Artículo 2 de la Ordenanza Nº 590/2011, se brindan cuatro conceptos claves para la normativa.

El primero de ellos es Software Libre que es entendido como \textbf{Programa de computación cuya licencia garantiza al usuario acceso al código fuente del programa y lo autoriza a ejecutarlo con cualquier propósito, modificarlo y redistribuir tanto el programa original como sus modificaciones en las mismas condiciones de licenciamiento acordadas al programa original, sin tener que pagar regalías a los desarrolladores previos}.

El segundo es Software de Código Abierto, definido como \textbf{Programa de computación cuya licencia garantiza al usuario acceso al código fuente del programa y lo autoriza a ejecutarlo con cualquier propósito, modificarlo y redistribuir tanto el programa original como sus modificaciones en las mismas condiciones de licenciamiento acordadas al programa original, sin imponer restricciones en otro software que distribuya junto con el mismo}.

El tercero es Software Privativo, entendido como \textbf{Programa de computación cuya licencia establece restricciones de uso, redistribución o modificación por parte de los usuarios, o requiere de autorización expresa del Licenciador}.

El cuarto es Estándares Abiertos, definidos como \textbf{Especificaciones técnicas, publicadas y controladas por alguna organización que se encarga de su desarrollo, las cuales han sido aceptadas por la industria, estando a disposición de cualquier usuario para ser implementadas en un software libre u otro, promoviendo la competitividad, interoperatividad o flexibilidad}. 

Por último, en el Artículo 3 se recomienda el empleo prioritario del software que garantice las libertades de ejecución, modificación y distribución con estándares abiertos en todas las actividades de la Universidad del Comahue, adoptando siempre que haya una alternativa el siguiente orden de prioridad:

a) Software Libre.
b) Software de Código Abierto.
c) Software privativo que respete los estándares abiertos.


\textbf{Considerandos}

En la Ordenanza se destaca que la Universidad Nacional del Comahue es depositaria de datos e información generada por la propia administración, miembros de la comunidad universitaria, otras instituciones universitarias, instituciones gubernamentales, organizaciones del tercer sector, empresas y ciudadanía en general.

Asimismo se considera que es responsabilidad y obligación de la gestión de gobierno controlar la seguridad, confiabilidad e interoperabilidad de la información que recibe, procesa y remite.

El empleo de formatos cerrados genera una dependencia tecnológica interminable hacia el proveedor de turno haciendo que el propio generador de la información requiera subordinarse a una aplicación sobre la que no tiene control para acceder a sus propios datos, por lo cual es necesario que se termine con la misma, implementando sistemas que permitan mantenerse en el mundo informático sin necesidad de depender de un proveedor.

En este sentido, el Software Libre permite la implementación de sistemas operativos, formatos y aplicaciones que podrán ser libremente utilizados y modificados cuando las necesidades lo requieran.

Actualmente, existen programas  pueden reemplazar sus respectivos pares de software privativo para todas las actividades administrativas que realiza la Institución educativa.

De hecho, en la UNCo hay capacidad técnica para llevar esto a cabo.

Cabe destacar que ya se han generado iniciativas en algunas Unidades Académicas y dependencias de esta universidad y de otras universidades nacionales y que el Grupo de Trabajo sobre Tecnologías de la Información y la Comunicación de la UNCo, conformado por la Secretaría General, la DTI, la UAI y la Facultad de Informática avala la propuesta.

Además, desde el Rectorado se estableció e implementó un programa de capacitación y migración al Software libre.

Por último, se da cuenta en la Ordenanza que es necesario establecer un marco jurídico para fijar políticas en el área informática.

}
%------------------------------------------------------------

\end{document}