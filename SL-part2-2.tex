%----------------------------------------------------------------------
%----------------------------------------------------------------------
%			2 ASPECTOS TECNICOS
%----------------------------------------------------------------------
%----------------------------------------------------------------------
\nota {	\item Aspectos técnicos. Proyectos de Software Libre. Modelo de desarrollo. Infraestructura tecnológica. Manejo de documentación y soporte. Seguridad.
}
 
\section{Aspectos técnicos}
\subsection {Código fuente y ejecutables} 
\begin{itemize}
	\item Lenguajes de programación
	\subitem Lenguajes compilados
	\subitem Lenguajes interpretados
	\item Formatos de archivos y documentos
	\subitem Archivos legibles por humanos
	\subsubitem Documentación: READMEs, Markup Languages, .pod, TeX/LaTeX, .rst...
	\subsubitem Configuración o datos: XML, YAML, .cfg,...
	\subitem Archivos estructurados
	\subsubitem Formatos abiertos y propietarios
	\subsubitem Formato .DOC, formato .odt
	\item Protocolos e interfaces
	\subitem Protocolos de Internet vs. protocolos propietarios
	\subitem RFCs 
	\item Bibliotecas
	\subitem Linking o vinculación
	\subitem Linking dinámico, Shared Objects (.so), DLLs
\end{itemize}

\subsection {Ciclo de compilación}
\figura[10]{ciclo}{Ciclo de compilación y ejecución}{ciclo_de_compilacion.eps}

\figura[8]{fuente}{Código fuente en lenguaje C}{bibliotecas-src-0.eps}

\figura{objeto}{Código objeto procedente del programa en C}{objeto.png}


\figura[8]{bibsrc1}{Un archivo fuente que usa dos funciones externas}{bibliotecas-src-1.eps}

\figura[8]{bibsrc2}{Archivo fuente conteniendo funciones}{bibliotecas-src-2.eps}

\figura[10]{bib1}{Compilación de los archivos fuente generando módulos objeto}{bibliotecas-1.eps}

\figura[10]{bib2}{Vinculación estática de objetos generando un ejecutable}{bibliotecas-2.eps}

\figura[10]{bib3}{Creación de una biblioteca compartida (\emph{shared object})}{bibliotecas-3.eps}

\figura[8]{bib4}{Uso de bibliotecas de vinculación dinámica}{bibliotecas-4.eps}

\figura[10]{bib5}{Compartiendo funciones de biblioteca}{bibliotecas-5.eps}

\subsection {Proyectos de Software Libre}
\begin{itemize}
	\item Motivación
	\item Perfil de los desarrolladores
	\item Modelo de desarrollo
	\item Comunidad y vitalidad
\end{itemize}

\subsubsection{La Catedral y el Bazar }
\begin{itemize}
	\item La Catedral y el Bazar vs. la Ingeniería de Software
	\item Representa la postura del movimiento Open Source
	\item Historia informal y razonada de un proyecto de SL
	\item Dos modelos de desarrollo
	\item Motivación y equipo de desarrollo inicial 
	\item Reutilizar antes que desarrollar desde cero
	\item Roles de los usuarios
	\item Liberar rápido y con frecuencia
\end{itemize}

\begin{tabular}{c|c|c}
 & Catedral & Bazar  \\
\hline
\hline
Gobierno & Vertical & Democrático \\
\hline
Roles 	 & Asignados & Libres \\
		 & Estáticos & Móviles \\
\hline
Usuarios & Clientes & Co-desarrolladores \\
\hline 
Procesos & Definidos & Multiestratégicos \\
\hline 
Entregas & Planificadas  	& \quotes{Cuando esté listo} \\
		 & Poco frecuentes 	& \emph{Release early}, \\
		 &				 	& \emph{Release often} \\
\hline
\end{tabular}

% --------------------------------------------------------------



\nota {Desarrollo abierto y colaborativo}

\subsection{Infraestructura tecnológica}
\begin{itemize}
	\item Internet y RFCs
	\item Freecode \url{http://freecode.com}
	\item Sourceforge \url{http://sourceforge.net}
	\item Google Code \url{http://code.google.com}
	\item GitHub \url{http://github.com}
\end{itemize}


\figura{freecode}{Nube de tópicos (\emph{tags}) de Freecode.org}{freecode.png}

\figura{github}{Interfaz del repositorio Github}{github.png}

\figura{googlecode}{Interfaz del repositorio Google Code}{googlecode.png}

\figura{centos}{Interfaz de edición del wiki de una distribución GNU/Linux}{centos.png}

\nota{
\subsection {Documentación y soporte}
\nota{CentOS,GLPI,Bandwidth Arbitrator,mosshe,tcng,zim,Libre Office }
\nota{Libro libre de Redes Olivier Bonaventure}
\subsection {Seguridad}
}



\subsection {Preguntas}

\begin{enumerate}
	\item ¿Cómo se pueden resumir las diferencias entre \emph{la catedral} y \emph{el bazar}?
	\item ¿Por qué en \emph{La Catedral y El Bazar} se hace énfasis en la capacidad de reutilizar código? ¿Qué relación especial tiene esta capacidad con el SL? 
	\item ¿Cómo suele tener origen un proyecto de SL? ¿Cuál suele ser el disparador o motivador de la creación de un proyecto nuevo? ¿En qué casos se considera necesario comenzar un proyecto nuevo en lugar de aprovechar proyectos existentes?
	\item ¿Qué roles asigna el modelo del Bazar a los usuarios del software?
	\item ¿A qué se llama un \emph{dictador benevolente}?
	\item ¿Por qué el modelo del Bazar recomienda la regla \quotes{liberar rápido y con frecuencia}?
	\item ¿En qué consiste la fase de captura de requerimientos de un proyecto de software? ¿En qué momento de la vida del proyecto se da esta fase?
	\item ¿A través de qué herramientas capturan requerimientos los proyectos de SL que usted conozca, en especial, los que su grupo está considerando para los trabajos prácticos de la materia?
	\item ¿Qué herramientas de comunicación suelen utilizar los desarrolladores de SL? ¿Qué funciones cumplen estas herramientas (de comunicación interpersonal, de transmisión de conocimiento técnico, de transmisión o análisis de código, de descripción de procesos...)?
	\item ¿En qué consiste la Ley de Brooks y por qué se dice que la complejidad de la interacción entre los desarrolladores se incrementa cuadráticamente? 
	\item Según Raymond, ¿de qué modo Internet y el SL se oponen a la Ley de Brooks? ¿A qué se llama el efecto Delphi?
	\item ¿Qué crítica se hace a esta última afirmación de Raymond, según el material \emph{Introducción al Software Libre} de Barahona y otros? 
	\item ¿Cuáles son los rasgos que hacen a los mejores líderes de proyecto, según Raymond?
\end{enumerate}


