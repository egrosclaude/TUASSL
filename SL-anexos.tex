
\section {Licencias libres y no libres}
\subsection{Comparación de GPL y MS EULA}

\subitem{EULA: End User License Agreement
\subitem{GPL: General Public License}
\subitem \url{http://blog.desdelinux.net/eula-windows-vs-gpl-linux-duelo-de-licencias}

\subsubsection{Características básicas de la EULA}
\shade{
Prohibida su copia y redistribución (copyright).
Puede ser usada por una sola computadora con un máximo de dos procesadores.
No puede ser utilizado como servidor web o como servidor de archivos.
Requiere registro después de 30 días.
Podría dejar de funcionar si se realizan cambios de hardware.
Las actualizaciones pueden cambiar la EULA si la compañía así lo decidiera.
Puede ser transferida al nuevo usuario una sola vez. El nuevo usuario debe estar de acuerdo con los términos de uso (EULA).
Impone limitaciones a la reingeniería inversa.
Se conceden permisos a Microsoft para tomar información sobre el Sistema y su uso.
Se conceden permisos a Microsoft para proveer esta información a otras organizaciones.
Se conceden permisos a Microsoft a realizar cambios a el sistema sin el consentimiento del usuario.
Garantía por los primeros 90 días, Actualizaciones, reparaciones y parches no tienen garantía. 
}
\subsubsection {Características básicas de la licencia libre GPL}
\shade {
Libertad de copiar, modificar y redistribuir el software.
Impide que un grupo o ente impida que otro grupo o ente no pueda tener estas mismas libertades.
Provee cobertura a los derechos de los usuarios de copiar, modificar y redistribuir el software.
Se puede vender si el usuario así lo decide, y los servicios conexos a dicho software pueden ser cobrados. 
Toda patente debe ser licenciada para el uso de todos o no ser licenciada en absoluto.
Software modificado no debe llevar costo de licencias.
Se debe proveer con el código fuente.
Si hay un cambio en la licencia, los términos generales de la licencia existente se mantienen. 
}
\section {Política de uso de SL en la UNC}

\subsubsection{Establecen la política de uso de Software Libre en la UNCo}

\emph {Martes, 13 de Diciembre de 2011 12:34}

\emph {Última actualización el Martes, 13 de Diciembre de 2011 20:10 }

 \emph {Escrito por Prensa UNCo}

Mediante la Ordenanza Nº 590 del 13 de diciembre de 2011, el Consejo Superior de la Universidad Nacional del Comahue resolvió establecer como política en el ámbito administrativo de la Casa de Altos Estudios el uso de Software Libre desarrollado con estándares abiertos en todos sus sistemas y equipamientos informáticos.

La medida fue tomada luego de que la Facultad de Informática elevara una solicitud en ese sentido.

 


En el Artículo 2 de la Ordenanza Nº 590/2011, se brindan cuatro conceptos claves para la normativa.

El primero de ellos es Software Libre que es entendido como \textbf{Programa de computación cuya licencia garantiza al usuario acceso al código fuente del programa y lo autoriza a ejecutarlo con cualquier propósito, modificarlo y redistribuir tanto el programa original como sus modificaciones en las mismas condiciones de licenciamiento acordadas al programa original, sin tener que pagar regalías a los desarrolladores previos}.

El segundo es Software de Código Abierto, definido como \textbf{Programa de computación cuya licencia garantiza al usuario acceso al código fuente del programa y lo autoriza a ejecutarlo con cualquier propósito, modificarlo y redistribuir tanto el programa original como sus modificaciones en las mismas condiciones de licenciamiento acordadas al programa original, sin imponer restricciones en otro software que distribuya junto con el mismo}.

El tercero es Software Privativo, entendido como \textbf{Programa de computación cuya licencia establece restricciones de uso, redistribución o modificación por parte de los usuarios, o requiere de autorización expresa del Licenciador}.

El cuarto es Estándares Abiertos, definidos como \textbf{Especificaciones técnicas, publicadas y controladas por alguna organización que se encarga de su desarrollo, las cuales han sido aceptadas por la industria, estando a disposición de cualquier usuario para ser implementadas en un software libre u otro, promoviendo la competitividad, interoperatividad o flexibilidad}. 

Por último, en el Artículo 3 se recomienda el empleo prioritario del software que garantice las libertades de ejecución, modificación y distribución con estándares abiertos en todas las actividades de la Universidad del Comahue, adoptando siempre que haya una alternativa el siguiente orden de prioridad:

a) Software Libre.
b) Software de Código Abierto.
c) Software privativo que respete los estándares abiertos.


\textbf{Considerandos}

En la Ordenanza se destaca que la Universidad Nacional del Comahue es depositaria de datos e información generada por la propia administración, miembros de la comunidad universitaria, otras instituciones universitarias, instituciones gubernamentales, organizaciones del tercer sector, empresas y ciudadanía en general.

Asimismo se considera que es responsabilidad y obligación de la gestión de gobierno controlar la seguridad, confiabilidad e interoperabilidad de la información que recibe, procesa y remite.

El empleo de formatos cerrados genera una dependencia tecnológica interminable hacia el proveedor de turno haciendo que el propio generador de la información requiera subordinarse a una aplicación sobre la que no tiene control para acceder a sus propios datos, por lo cual es necesario que se termine con la misma, implementando sistemas que permitan mantenerse en el mundo informático sin necesidad de depender de un proveedor.

En este sentido, el Software Libre permite la implementación de sistemas operativos, formatos y aplicaciones que podrán ser libremente utilizados y modificados cuando las necesidades lo requieran.

Actualmente, existen programas  pueden reemplazar sus respectivos pares de software privativo para todas las actividades administrativas que realiza la Institución educativa.

De hecho, en la UNCo hay capacidad técnica para llevar esto a cabo.

Cabe destacar que ya se han generado iniciativas en algunas Unidades Académicas y dependencias de esta universidad y de otras universidades nacionales y que el Grupo de Trabajo sobre Tecnologías de la Información y la Comunicación de la UNCo, conformado por la Secretaría General, la DTI, la UAI y la Facultad de Informática avala la propuesta.

Además, desde el Rectorado se estableció e implementó un programa de capacitación y migración al Software libre.

Por último, se da cuenta en la Ordenanza que es necesario establecer un marco jurídico para fijar políticas en el área informática.

}

\section {Cuestionario para identificación de riesgos operativos}
\label{sec:CuestionarioRiesgos}
Tomado de \textbf{A Taxonomy of Operational Risks, CMU/SEI-2005-TN-036} - \url{http://www.sei.cmu.edu/reports/05tn036.pdf}

\subsection{Taxonomía de Riesgos}

\subsubsection{A - Misión}
Riesgos a la operación que pueden surgir a causa de la naturaleza de la misión que se intenta cumplir.

\textbf{1. Tareas, órdenes, planes}

¿Hay riesgos que puedan surgir de la forma en que se  distribuyen las tareas, en que se dan las órdenes, o en la que se desarrollan los planes operativos? (Estabilidad, completitud, claridad, validez, factibilidad, precedencia, oportunidad)

\textbf{2. Ejecución de la misión}

¿Hay riesgos que puedan surgir de la ejecución de la misión? (Eficiencia, efectividad, complejidad, oportunidad, seguridad)

\textbf{3. Producto o servicio}

¿Hay riesgos que puedan surgir del producto final o servicio que provee esta misión operativa? (Usabilidad, efectividad, oportunidad, exactitud, correctitud) 

\textbf{4. Sistemas operativos}

¿Hay riesgos que puedan surgir de los sistemas operativos usados? (Productividad, adecuación, usabilidad, familiaridad, confiabilidad, seguridad, inventario, instalaciones, soporte de sistemas)

¿Hay otros riesgos que puedan surgir de la misión, que no estén cubiertos por las categorías anteriores?


\subsubsection{B - Procesos}
Riesgos a la misión que puedan surgir de la forma en que la organización ejecuta la misión.

\textbf{1. Procesos operativos}

¿Hay riesgos que puedan surgir del proceso que ha elegido la organización operativa para ejecutar la misión? (Formalidad, adecuación, control de procesos, familiaridad, control de producto)

\textbf{2. Procesos de mantenimiento}

¿Hay riesgos que puedan surgir del proceso que utiliza la organización de mantenimiento para mantener los sistemas operativos? (Formalidad, adecuación, control de procesos, familiaridad, calidad de servicio)

\textbf{3. Procesos de administración}

¿Hay riesgos que puedan surgir de la forma en que se planifica, monitoriza o controla el presupuesto operativo, o en la estructura operativa de la organización, o en su forma de manejar interfaces internas y externas? (Planeamiento, organización, experiencia en administración, interfaces de los programas)

\textbf{4. Métodos de administración}

¿Hay riesgos que puedan surgir de la forma en que se maneja el personal operativo? (Monitorización, administración de personal, aseguramiento de calidad, administración de configuración)

\textbf{5. Ambiente de trabajo}

¿Hay riesgos que puedan surgir del ambiente general, o de la organización a la cual pertenece la unidad operativa? (Actitud hacia la calidad, cooperación, comunicación, moral)

¿Hay otros riesgos que puedan surgir de la forma en que la unidad operativa enfrenta la misión, pero que no estén cubiertos por las categorías mencionadas?

\subsubsection{C - Restricciones}
Riesgos a la misión que puedan surgir de orígenes fuera de nuestro control.

\textbf{1. Recursos}

¿Hay riesgos que puedan surgir de recursos que necesite la organización operativa pero que esté fuera de su control obtener o mantener? (Agenda, personal, presupuesto, instalaciones, herramientas)

\textbf{2. Políticas}

¿Hay riesgos que puedan surgir de políticas legalmente vinculantes o restrictivas? (Leyes y reglamentaciones, restricciones, obligaciones contractuales)

\textbf{3. Interfaces}

¿Hay riesgos que puedan surgir de interfaces externas que la organización operativa no pueda controlar? (Comunidad de usuarios o clientes, agencias asociadas, contratistas, líderes mayores, vendedores, políticas)

¿Hay otros riesgos que puedan surgir de factores fuera del control de la organización operativa, pero que no estén cubiertos por las categorías mencionadas?