%----------------------------------------------------------------------
%----------------------------------------------------------------------
%----------------------------------------------------------------------
%----------------------------------------------------------------------
%			4	USO Y APLICACION
%----------------------------------------------------------------------
%----------------------------------------------------------------------
%----------------------------------------------------------------------
%----------------------------------------------------------------------
\nota {
	\section{Implantación de sistemas de Software Libre}
	\item Uso y aplicación de Software Libre. Costo total de operación. Estudio de costo/beneficio. Procesos de migración. 
}

\section{Uso y aplicación de Software Libre}

\nota{
\subsection{Aspectos económicos}
\begin{itemize}
	\item Productos, Soluciones y Servicios
	\item Modelos de negocio\footnote{\url{http://www.fsfla.org/~lxoliva/papers/free-software/beautiful-mind.pdf}}
 

	\item Proceso de migración
	\item Costo total de operación
\end{itemize}
\subsection{SL en el sector público}
\label{sub:SLenelsectorpúblico}
}

\subsection {Selección de Software Libre}
\begin{itemize}
	\item Cualidades del software
		\subitem Proyecto vivo y en actividad
		\subitem Proyecto más aceptado
		\subitem Estable y maduro
		\subitem Tendencia
	\item Cualidades de la organización
		\subitem Impacto que provocará el cambio
		\subitem Estudio de Costos/Beneficios 
		\subitem ${C_1/B_1 \longleftrightarrow C_2/B_2}$
		\subitem Estudio de Costo Total de Propiedad de ambas soluciones
		\subitem Riesgos
		\subsubitem Taxonomía de riesgos (ver Anexo \ref{sec:CuestionarioRiesgos})
	\item Valuación de costos, beneficios, riesgos
	\subitem Difícil en forma absoluta
	\subitem Cuando se trata de comparar se puede fijar una escala adimensional
	\subitem Los riesgos se pueden computar como ${probabilidad de ocurrencia * gravedad}$
\end{itemize}

\subsection {Administración de TI}
\begin{itemize}
	\item Frameworks de administración de Tecnologías de Información (TI)
	\item Compendios de mejores prácticas
		\subitem CobiT \url{}
			\subsubitem TI en general
			\subsubitem Libremente reproducible
		\subitem ITIL \url{}
			\subsubitem Servicios
			\subsubitem Más detallada
			\subsubitem Licencia restrictiva
		\subitem ISO
\end{itemize}

\subsection {CobiT}
\begin{itemize}
	\item CobiT define 34 procesos de TI. Incluye herramientas para medición de desempeño, una taxonomía de factores críticos de éxito, y modelos de madurez para asistir en la toma de decisiones hacia el mejoramiento de capacidades. 
	\item Objetivos de CobiT
		\subitem Asegurar que 
		\subsubitem TI se alinea con la actividad de la organización
		\subsubitem TI habilita las operaciones y maximiza los beneficios
		\subsubitem Los recursos se usen responsablemente
		\subsubitem Los riesgos se manejen adecuadamente
	\item Dominios definidos en CobiT
		\subitem Plan and Organise (PO)
		\subitem Acquire and Implement (AI)
		\subitem Deliver and Support (DS)
		\subitem Monitor and Evaluate (ME)
	\emph {
		\subitem \quotes{Acquire and Implement, AI}
		\subsubitem Identify Automated Solutions
		\subsubitem Acquire and Maintain Application Software
		\subsubitem Acquire and Maintain Technology Infrastructure
		\subsubitem Enable Operation and Use
		\subsubitem Procure IT Resources
		\subsubitem Manage Changes
		\subsubitem Install and Accredit Solutions and Changes
	}
\end{itemize}

\figura {cobit}{Un ejemplo de guía de actividades de CobIT}{CobIT-RACI.png}





\subsection {Migraciones en general}
\begin{itemize}
	\item Difícil por la naturaleza de los sistemas privativos
	\item Apoyo de los usuarios
	\subitem Se necesitan defensores del cambio
	\subitem Información para involucrarlos
	\item Diagnosticar claramente el escenario de partida
	\subitem Hardware, software, configuración de la red, usuarios
	\item Definir la situación final deseada
	\item Justificar la migración
	\subitem Análisis de costos de la migración y proyección de ahorros factibles
	\item Planificar la migración
	\item Migración de los datos
	\item Preparar y capacitar el equipo técnico
\end{itemize}

\subsection {Formas de migración a SL}
\begin{itemize}
	\item Repentina
		\subitem Riesgosa
		\subitem Requiere gran inversión en capacitación
	\item Progresiva
	\begin{enumerate}
		\item Usando SL bajo el sistema operativo existente
		\item Migrando los datos progresivamente
		\item Migrando los ambientes y sistemas operativos	
	\end{enumerate}
\end{itemize}

\subsection {Facilitar migración futura}
\begin{enumerate}
	\item Usar Formatos y estándares abiertos
	\item Usar metodologías basadas en capas para el desarrollo, separando código de interfaz y de acceso a los datos 
	\item Generar aplicaciones portables, evitando lenguajes de arquitecturas específicas
	\item Evitar la construcción de aplicaciones que requieran otras privativas
	\item Desarrollos web que cumplan estándares W3C y validados contra navegadores libres
\end{enumerate}

\subsection {Costo Total de Propiedad, CTP}
\begin{itemize}
	\item O también \emph{Total Cost of Ownership, TCO}
	\item Cálculo para facilitar la evaluación de costos directos e indirectos asociados con la adopción de un componente de IT 
	\item Evaluar beneficios de la migración frente a los costos
	\item El CTP define el costo total de la propiedad para el uso de una tecnología concreta durante el período de vida de dicha tecnología
	\subitem Medición del impacto
	\item Taxonomías de costos
	\subitem Costos de adquisición
	\subitem Costos asociados con reparaciones
	\subitem Costos de oportunidad
	\subitem Costos extendidos
	\subsubitem Servicios conexos de soporte, redes, seguridad, capacitación, licenciamiento de software u otros componentes
	\item ¿Taxonomías de \emph{beneficios}?
\end{itemize}

\subsection {Metodologías de análisis de CTP}
\begin{itemize}
	\item Una metodología genérica
	\subitem \url{http://blogs.msdn.com/b/eduardop/archive/2006/05/29/610441.aspx}
	\subitem Considera costos divididos en iniciales y operativos
	\subitem Costo inicial de la solución ${CI = CH + CS + CINS + CCON}$
	\subsubitem CH = Costo del hardware
	\subsubitem CS = Costo del software
	\subsubitem CINS = Costo de servicios iniciales de instalación
	\subsubitem CCON = Costo de servicios iniciales de configuración

	\subitem Costo de administración (CA)
	\subsubitem Personal dedicado al mantenimiento operativo
	\subsubitem Cálculos basados en honorarios y gastos por personal

	\subitem Costo de operación (CO)
	\subsubitem Pérdidas por caída de operación o soporte reactivo
	\subsubitem Cálculos basados en costo de incidentes

	\subitem Costo de soporte (CS)
	\subsubitem Costo de proveer asistencia a usuarios
	\subsubitem Basado en costo por hora de soporte
	
	\subitem TV = Tiempo de vida de la solución
	\subitem Costo total ${= CI + (CA + CO + CS) * TV}$


	\item Otra metodología orientada a SL
	\subitem Centro de Excelencia en Software Libre U. Castilla-La Mancha
	\subitem \url {http://www.bilib.es/documentos/Taller_de_Migracion.pdf}
	\subitem Costo Total CT = CD + CI = Costos Directos + Costos Indirectos
	\subsubitem Directos se producen como consecuencia de la adquisición del software
	\subsubitem Indirectos se producen como consecuencia de pérdidas o caídas de productividad
	\subitem ${CD = CH + CS + CSOP + CFOR + CPER}$
	\subsubitem CH = Costo del hardware
	\subsubitem CS = Costo de licenciamiento de software
	\subsubitem CSOP = Costo de soporte = Instalación + Configuración + Mantenimiento
	\subsubitem CFOR = Costo de formación del personal de operación
	\subsubitem CPER = Costo de personal de operación
	\subitem ${CI = CM + CC + CSEG + CE + CDISP}$
	\subsubitem CM = Costo de mantenimiento por errores o problemas del software
	\subsubitem CC = Costo de oportunidad debido a caídas de sistema
	\subsubitem CSEG = Costo de aseguramiento informático
	\subsubitem CE = Costo de asegurar la escalabilidad
	\subsubitem CDISP = Costo de asegurar la disponibilidad

\end{itemize}