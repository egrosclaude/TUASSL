%----------------------------------------------------------------------
%----------------------------------------------------------------------
%----------------------------------------------------------------------
%			3 	ASPECTOS LEGALES
%----------------------------------------------------------------------
%----------------------------------------------------------------------
%----------------------------------------------------------------------
\nota {
	\item Aspectos legales. Dominio Público, Copyright, Copyleft y Licenciamiento. Licencias de FSF, Creative Commons, OSI, otras. 
}



\section{Aspectos legales}
\nota {Ética hacker - \url{http://es.wikipedia.org/wiki/Partido_Pirata} - Torrent

\url{http://www.cultofmac.com/221474/hackers-can-grab-control-of-your-camera-via-its-wi-fi-sd-card}
}
\figura{categorias}{Categorías de software según FSF}{categorias.png}

\subsection{Las licencias de software}
\begin{itemize}
	\item Código, código fuente, código objeto, código ejecutable
	\item Documentación fuente y en otros formatos
	\item Dominio Público
	\item El concepto de Copyleft
	\item Las licencias de FSF
	\item The Open Source Definition
	\subitem \url{http://opensource.org/docs/osd}
	\item Creative Commons
	\subitem \url{http://www.creativecommons.org.ar/}
\end{itemize}



\figura[4]{copyleft}{Símbolo de Copyleft}{copyleft.png}
\figura[6]{commons}{Logotipo de Creative Commons}{CC-logo.png}

\subsection{Preguntas}
\begin{enumerate}
	\item El acceso al código fuente, ¿implica automáticamente la posibilidad de introducir modificaciones y mejoras?  ¿Y la de redistribuir el software? 
	\item Si modifico un software con licencia libre, ¿estoy obligado por la licencia libre a redistribuirlo públicamente? 
	\item En caso de redistribuir software liberado bajo licencia libre, ¿a qué me obliga una licencia libre? ¿A qué se obliga la persona que lo recibe de mí?
	\item Si descargo un software publicado bajo una licencia libre, ¿puedo redistribuirlo? ¿Puedo cobrar por redistribuirlo? ¿Y si se trata de una licencia Open Source?
	\item Si descargo un software publicado bajo una licencia libre, ¿puedo modificarlo? ¿Puedo cobrar por modificarlo? ¿Y si se trata de una licencia Open Source?
	\item ¿En qué consisten las licencias de software permisivas y qué ejemplos se pueden dar?
	\item ¿Cuál es la opinión de FSF respecto del movimiento Open Source?
	\item ¿Cómo puede resumirse la diferencia entre los conceptos de Free Software y Open Source?
	\item Mencione proyectos o productos Open Source que no sean considerados libres por FSF.
	\item ¿Qué licencias reconoce y no reconoce FSF?
	\item ¿Cuál ha sido la historia de versiones de GPL hasta el momento y qué motivó cada versión? ¿Considera que puede ser necesaria una nueva versión en algún momento futuro?
	\item ¿A qué se llama en inglés \emph{appliances}? ¿Qué ejemplos conoce? 
	\item ¿Qué es el producto Tivo y de qué manera se relaciona con la historia de la licencia GPL?
	\item ¿Cuándo un dispositivo \emph{appliance} respeta las libertades, según FSF?
	\item ¿Puede identificar desarrollos originalmente libres que fueron absorbidos por intereses corporativos? ¿Qué licencias utilizaron?
	\item ¿Puede mencionar dispositivos \emph{appliance} que respeten y no respeten las libertades según FSF?

	\item ¿Qué significa \emph{jailbreak} y cuál es la posición legal de las empresas al respecto? ¿Qué es 'rooting'? ¿Qué significa 'ingeniería inversa'? ¿Es una actividad legal? 

	\item ¿Qué significa DRM? ¿De qué manera afecta DRM a las libertades del usuario sgún FSF?

\end{enumerate}
