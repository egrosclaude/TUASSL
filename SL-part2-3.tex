%----------------------------------------------------------------------
%----------------------------------------------------------------------
%----------------------------------------------------------------------
%			3 	ASPECTOS LEGALES
%----------------------------------------------------------------------
%----------------------------------------------------------------------
%----------------------------------------------------------------------
\nota {
	\item Aspectos legales. Dominio Público, Copyright, Copyleft y Licenciamiento. Licencias de FSF, Creative Commons, OSI, otras. 
}



\section{Aspectos legales}
\nota {Ética hacker - \url{http://es.wikipedia.org/wiki/Partido_Pirata} - Torrent
		\url{http://www.cultofmac.com/221474/hackers-can-grab-control-of-your-camera-via-its-wi-fi-sd-card}
}


\subsection {Propiedad intelectual}

\begin{itemize}
	\item Copyright o derechos de autor
	\subitem Comprenden la expresión de un contenido, no el contenido
	\subitem Protección sobre la copia no autorizada
	\subitem Aparece al momento de publicación
	\item Dominio público: no existe detentor de los derechos
	\item Derechos patrimoniales
	\subitem Expiran en un cierto plazo luego de la muerte del autor
	\subitem La obra pasa al dominio público
	\item Derechos morales
	\subitem Son permanentes
	\item Propiedades protegidas
	\subitem Obras literarias (software, caso especial), imágenes, filmaciones, composiciones musicales, otras
\end{itemize}


\subsection {Licencias}
\begin{itemize}
	\item Cesión de derechos = licencia
	\item Contrato entre autor y usuarios
	\item Firma del contrato = instalar el producto
	\item Cesión de derechos de uso, nunca de propiedad
	\item Licencias restrictivas
	\subitem    Aseguran derechos del autor
	\subitem    Restringen derechos del usuario
	\item Licencias libres
	\subitem   Aseguran derechos del usuario
\end{itemize}





\begin{itemize}
	\item Secreto comercial
		\subitem  Industria química, de alimentos
		\subitem  Ingeniería inversa, a veces prohibida
	\item Patentes
	\subitem   Una innovación es revelada públicamente

	\subitem  Haciendo posible su reproducción
 	\subitem   A cambio se obtiene un monopolio temporario
 		\subitem Promueve la investigación privada, distribuyendo el producto de esa investigación y sin costos para el contribuyente
 		\subitem Pero puede dar lugar a una industria del juicio, parásita e improductiva
 		\item Definición de innovación
 		\subitem     Inventos
 		\subitem    Algoritmos
 		\subitem    Programas
 		\subitem    Modelos de negocio
 		\subitem    Sustancias naturales
 		\subitem    Componentes de la vida
	\item Cultura Libre
	\subitem \url{http://es.wikipedia.org/wiki/Cultura_libre}

\end{itemize}

\subsection{Licencias de software}
\begin{itemize}
	\item Código, código fuente, código objeto, código ejecutable
	\item Documentación fuente y en otros formatos
	\item Concepto de Copyleft
	\subitem \url {http://es.wikipedia.org/wiki/Copyleft}
	\subitem Permitir la libre distribución de copias y versiones modificadas de una obra, exigiendo que en esas versiones se preserven los mismos derechos de uso
	\item Las licencias de FSF
	\subitem \url {http://es.wikipedia.org/wiki/GNU_General_Public_License}
	\subitem \url{http://www.gnu.org/licenses/gpl-faq.es.html}
	\item The Open Source Definition, de OSI
	\subitem Licencias Protectivas y No Protectivas
	\subitem \url{http://es.wikipedia.org/wiki/Licencia_de_código_abierto}
	\subitem \url{http://opensource.org/docs/osd}
\end {itemize}


\subsection {Licencias FSF}
\subsubsection {GPLv1}
Permite la modificación y libre distribución de copias y requiere que se preserven las mismas libertades en las obras derivadas. La distribución del original o de las obras derivadas no debe impedir el acceso a los fuentes. Otras licencias bajo las cuales estén las obras no deben restringir cláusulas de la GPL.

\subsubsection {LGPL}

Con menos restricciones que GPL. Permite distribuir obras derivadas que se vinculan con software libre o propietario, como bibliotecas o drivers. 

\subsubsection {GPLv2}

El cambio principal respecto de la GPLv1 consiste en que si el software tiene alguna restricción en conflicto con una parte protegida por GPL, no puede redistribuirse (\quotes{Libertad o Muerte}). 

\subsubsection {GPLv3}
(\quotes{Anti-tivoización} y anti-acuerdos de patentes). Impide que un distribuidor asocie software GPL a un dispositivo que no permite la modificación de dicho software. Impide que una obra sea entregada a un distribuidor que cobre por el trabajo del autor. Formaliza los permisos adicionales que conceden los autores. Modifica la terminación de la licencia por violación (simplifica el \quotes{pedir perdón} a los propietarios).

\begin{itemize}
	\item Tivoización
	\subitem Creación de un sistema que usa software protegido por copyleft pero incluye hardware que restringe la libertad de los usuarios de correr obras derivadas de ese software. Es ilegal bajo la 
GPLv3.

\end{itemize}

\subsubsection {Affero GPL o AGPL}

GPLv2 más cláusula que obliga a distribuir los fuentes del software si éste se utiliza ofreciendo servicios a través de una red.

\subsection {Licencias tipo BSD}
\begin{itemize}
	\item \quotes{Minimalistas}
	\item Originadas en proyectos de extensión de universidades, no garantizan la libertad del software

	\item Las redistribuciones en fuente o binarios deben preservar la nota de copyright
	\item No se puede usar el nombre del propietario para promocionar obras derivadas
	\item El propietario no se responsabiliza por el uso del programa
\end{itemize}



\figura{categorias}{Categorías de software según FSF}{categorias.png}



\subsection {Licencias de contenidos libres}
\begin{itemize}
	\item Creative Commons
	\item Dimensiones 
	\subitem Atribución
	\subitem Uso comercial
	\subitem Obras derivadas
	\subitem Compartir igual
	\item \url{http://www.creativecommons.org.ar/}
	\item Elegir una licencia: \url {http://creativecommons.org/choose/?lang=es}
\end{itemize}


\nota{
\figura[4]{copyleft}{Símbolo de Copyleft}{copyleft.png}
\figura[6]{commons}{Logotipo de Creative Commons}{CC-logo.png}
}

\begin{tabular}{c|c|c|c|c|c}
Licencia & Atribución 	& Uso comercial & Obras derivadas & Compartir igual &\\
\hline
\hline
{CC-BY} 		& \ding{108} & \ding{108}	& 	\ding{108}	& 	& \includegraphics[width=2cm]{img/cc-CC-BY.png}\\
\hline	
{CC-BY-NC} 	& \ding{108} &  			& 	\ding{108}	&  		& \includegraphics[width=2cm]{img/cc-CC-BY-NC.png}\\
\hline
CC-BY-NC-ND & \ding{108} &  			& 	 			&  		& \includegraphics[width=2cm]{img/cc-CC-BY-NC-ND.png}\\
\hline
CC-BY-NC-SA & \ding{108} &  			& 	\ding{108}	&  \ding{108}	& \includegraphics[width=2cm]{img/cc-CC-BY-NC-SA.png}\\
\hline
CC-BY-ND 	& \ding{108} &	\ding{108} 	& 				&  		& \includegraphics[width=2cm]{img/cc-CC-BY-ND.png}\\
\hline
CC-BY-SA 	& \ding{108} & \ding{108}	& 	\ding{108} 	&  \ding{108}	& \includegraphics[width=2cm]{img/cc-CC-BY-SA.png}\\
\hline
\end{tabular}



\subsection{Preguntas}
\begin{enumerate}
	\item El acceso al código fuente, ¿implica automáticamente la posibilidad de introducir modificaciones y mejoras?  ¿Y la posibilidad de redistribuir el software? 
	\item Si modifico un software con licencia libre, ¿estoy \emph{obligado} por la licencia libre a redistribuirlo públicamente? 
	\item ¿Cómo se define el concepto de Copyleft?
	\item Al licenciar una obra bajo alguna forma de Copyleft, ¿se pierde el Copyright?
	\item En caso de redistribuir software liberado bajo licencia libre, ¿a qué me obliga una licencia libre? ¿A qué se obliga la persona que lo recibe de mí?
	\item Si una empresa cobra por la distribución de software liberado bajo licencia libre, ¿está violando la licencia?
	\item Si descargo un software publicado bajo una licencia libre, ¿puedo redistribuirlo? ¿Puedo cobrar por redistribuirlo? ¿Y si se trata de una licencia Open Source?
	\item Si descargo un software publicado bajo una licencia libre, ¿puedo modificarlo? ¿Puedo cobrar por modificarlo? ¿Y si se trata de una licencia Open Source?
	\item ¿Cuál es la motivación de FSF para asumir el copyright de las obras libres?
	\item ¿En qué consisten las licencias de software \quotes{minimalistas} o permisivas y qué ejemplos se pueden dar?
	\item ¿Cuál es la opinión de FSF respecto del movimiento Open Source?
	\item ¿Cómo puede resumirse la diferencia entre los conceptos de Free Software y Open Source?
	\item Mencione proyectos o productos Open Source que no sean considerados libres por FSF.
	\item ¿Cuáles son las principales licencias que reconoce FSF como libres?
	\item ¿A qué se llama el \quotes{efecto viral} de la GPL?
	\item ¿Cuál ha sido la historia de versiones de GPL hasta el momento y qué motivó cada versión? ¿Considera que puede ser necesaria una nueva versión en algún momento futuro?
	\item ¿Por qué se creó la licencia LGPL? ¿En qué casos se recomienda usar o no usar la LGPL?
	\item ¿Por qué no es lo mismo decir Software No Libre que Software Comercial? ¿Por qué no es lo mismo Licencia Libre que Dominio Público? 
	\item ¿Qué es MP3? ¿Qué clase de protección legal tiene MP3 y qué consecuencias tiene para los usuarios? ¿Existen alternativas técnicamente equivalentes y legalmente menos restrictivas?
	\item ¿A qué se llama en inglés \emph{appliances}? ¿Qué ejemplos conoce? 
	\item ¿Qué es el producto Tivo y de qué manera se relaciona con la historia de la licencia GPL?
	\item ¿Cuándo un dispositivo \emph{appliance} respeta las libertades, según FSF?
	\item ¿Puede identificar desarrollos originalmente libres que fueron absorbidos por intereses corporativos generando software no libre? ¿Qué licencias utilizaron?
\item ¿Cuáles son las características de las licencias tipo BSD? ¿Qué proyectos de SL conocidos las emplean?
	\item ¿Puede mencionar dispositivos \emph{appliance} que respeten y no respeten las libertades según FSF?
	\item Si una empresa crea una modificación de un producto de SL y lo utiliza en forma interna, ¿debe publicar los fuentes de la obra derivada? ¿Y si ofrece servicios a través de la red utilizando esa obra derivada?
	\item ¿Qué significa \emph{jailbreak}? ¿Qué es \quotes{rooting}? ¿Qué significa \quotes{ingeniería inversa}? ¿Es una actividad legal? 
	\item ¿Qué significa DRM? ¿De qué manera afecta DRM a las libertades del usuario según FSF? \footnote{\url{http://es.wikipedia.org/wiki/Gestión_digital_de_derechos}}

	\item ¿Qué dimensiones comprenden las licencias Creative Commons? ¿Cuál de las dimensiones corresponde al concepto de Copyleft según FSF? ¿Qué licencias CC no incluyen Copyleft?

\item ¿Qué es WIPO? ¿Qué opinión sobre Creative Commons volcó la Argentina en su propuesta, conjunta con Brasil, a la WIPO en 2004? ¿A qué se refiere esta propuesta con \quotes{medidas tecnológicas de protección en el entorno digital}? ¿A qué se denomina \quotes{la dimensión del desarrollo}? \footnote{\url{http://www.wipo.int/edocs/mdocs/govbody/es/wo_ga_31/wo_ga_31_11.pdf}}

\item ¿Existen las patentes de SW en Argentina? \footnote{\url{http://www.vialibre.org.ar/mabi/4-software-patentado.htm}}.
¿Cuáles son los peligros de un sistema de patentes \quotes{permisivo}?

\end{enumerate}
