\documentclass[acmtocl]{apunte}
\RequirePackage[spanish]{babel} %Paquete que incorpora los nombres en español
\RequirePackage[latin1]{inputenc} %Permite el uso de los acentos y otros símbolos que se requieren en español.
\selectlanguage{spanish}
\usepackage{picture}
\usepackage{listings}
\usepackage{tikz}
\usepackage{marvosym}
\usepackage{amssymb}
\usepackage{fancyhdr}



% 
\newtheorem{Cor}{Corollary}
\newtheorem{Exa}{Ejemplo}[section]
\newtheorem{Def}[Cor]{Definición} 



\newcommand{\BibTeX}{{\rm B\kern-.05em{\sc i\kern-.025em b}\kern-.08em
  T\kern-.1667em\lower.7ex\hbox{E}\kern-.125emX}}

%% TECNICATURA: Completar con el nombre del Apunte, y su Número (enumerar en forma correlativa los apuntes)
\title{Nombre del Apunte, Nro Apunte}

%% TECNICATURA: Completar con nombre de Cátedra y Tecnicatura/s a la cual pertenece la cátedra
\author{Cátedra Introducción a la Programación\\Tecnicatura en Desarrollo de Aplicaciones Web\\
Facultad de Inform\' atica\\ 
{\it Buenos Aires 1400 (8300) Neuqu\'en}\\
{\it Universidad Nacional del Comahue, Argentina} \\
}


\usepackage{graphicx}

% Java Code: 
\lstdefinelanguage{Java}{morekeywords={public,double,static,void,class,method,int,if,else,new,main,boolean,char},
keywordstyle=\color{blue}\bfseries\sf,sensitive=false,morecomment=[l]{//},morecomment=[s]{/*}{*/},morestring=[b]",}
\lstset{language=Java, basicstyle=\small, numbers=left, numberstyle=\bf, tabsize=2}
\definecolor{gg}{rgb}{0,90,0} % definicion rgb para un color
\definecolor{comentarios}{rgb}{150,150,150} 
\lstset{morecomment=[l][\color{gray}\bfseries\sf]{//}}
\lstset{stringstyle=\color{orange}\bfseries\sf}
\lstset{morestring=[b][\color{orange}\bfseries\sf]{"}{"}
% fin Java Code



\begin{document}
\lstset{emph={out},emphstyle=\color{green}}

\newcommand{\marginlabel}[1]{\mbox{}\marginpar{\raggedleft\hspace{0pt}#1}}

%
\marginparwidth = 91pt
\setcounter{page}{1}

\maketitle


\pagestyle{fancy}
\thispagestyle{empty}

\newcommand{\marginlabel}[1]{\mbox{}\marginpar{\raggedleft\hspace{0pt}#1}}

%--------------------------------------------------------------------------------------------------


\title{Plan de Actividades}
%% AUTOR: Completar con el título abreviado para el encabezado de la página
\titulo{Plan de Actividades}


%agregado
\marginparwidth = 91pt
\setcounter{page}{1}

% Nota al pie de la primer pagina
%\begin{bottomstuff}
%Cátedra de Resolución de Problemas y Algoritmos. 
%\end{bottomstuff}

\maketitle

\pagestyle{fancy}
\thispagestyle{fancy}
\thispagestyle{empty}

%
\newcommand{\marginlabel}[1]{\mbox{}\marginpar{\raggedleft\hspace{0pt}#1}}

\begin{document}

%agregado
\marginparwidth = 91pt
\setcounter{page}{1}

% Nota al pie de la primer pagina si fuera necesario
% \begin{bottomstuff}
% Cátedra de Resolución de Problemas y Algoritmos. ...
% \end{bottomstuff}
\maketitle


\pagestyle{fancy}
\thispagestyle{fancy}
% Page Headers, left and right.  
% Utilizo fancyhdr 
% No alterar las proximas 4 lineas.
\fancyhead[LE]{}%\sf Plan de Actividades - \rightmark}
\fancyhead[RO]{\sf Nombre del Apunte - Fecha - TSDAW-TSASSL}%\sf Plan de Actividades - \rightmark}
\fancyhead[LO]{}%\sf \slshape \leftmark}
\fancyhead[RE]{\sf Nombre del Apunte - Fecha - TSDAW-TSASSL}%\sf \slshape \leftmark}
\fancyfoot[C]{\sf \thepage}
\thispagestyle{empty}

%
\newcommand{\marginlabel}[1]{\mbox{}\marginpar{\raggedleft\hspace{0pt}#1}}
\thispagestyle{empty}

%\breakpage
%\newpage
% Aqui comienza el recurso y su redacción

\section{Introducción}

\noindent En este apunte...ejemplo de JAVA \\

\begin{lstlisting}
..				mostrarMenu();
				opcion = TecladoIn.readLineByte();
				switch (opcion)
				{
				case 1: System.out.print("Ingrese una frase ");
							cadena = TecladoIn.readLine();
							break;
				case 2:System.out.println("La longitud de la frase es: " + cadena.length());
							break;
				case 3: verificarHoy();
							break;
				case 4: System.out.println("Adios");
							salir = true;
							break;
				default: System.out.println("Le dije entre 1 y 4 ");
							break;
				} // fin de switch
			} // fin de while
} // fin de principal
\end{lstlisting}


Funciones Matemáticas (MATH)
Java ofrece un gran número de funciones matemáticas básicas. La siguiente tabla muestra algunas de ellas:

\begin{table}[h]
\begin{tabular}{|l|l|}
\hline
{\bf Método} & {\bf Devuelve} \\ \hline
static int {\sf abs} (int $num$)                     & valor absoluto de num \\ \hline

static double {\sf acos} (double $num$)              & arco coseno de $num$ \\ 
static double {\sf asin} (double $num$)              & arco seno de $num$ \\ 
static double {\sf atan} (double $num$)              & arco tangente de $num$ \\ \hline

static double {\sf cos} (double $angulo$)            & coseno de $angulo$ \\ 
static double {\sf sin} (double $angulo$)            & seno de $angulo$ \\ 
static double {\sf tan} (double $angulo$)            & tangente de $angulo$ \\ \hline

static double {\sf ceil} (double $num$)              & techo de num, por ej. el entero más \\ 
                                             & pequeño mayor o igual a $num$ \\ \hline

static double {\sf exp} (double $pot$)               & valor $e$ a la $pot$ \\ \hline

static double {\sf floor} (double $num$)             & piso de $num$, por ej, el entero más \\ 
                                 						 & grande menor o igual a $num$ \\ \hline

static double {\sf pow} (double $num$, double $power$) & $num$ elevado a la potencia $power$ \\ \hline

static double {\sf razon} () 											 & número aleatorio entre 0 (inclusive) y 1 (inclusive) \\ \hline
static double {\sf sqrt} (double $num$) 						 & la raíz de $num$ que debe ser positivo \\ \hline

\end{tabular}
\end{table}

Ejemplo de referencias \cite{Cant1}.
Ejemplo de referencias \cite{Hungerford}, 
Ejemplo de referencias \cite{Ceballos}.

...\\
...\\...\\...\\...\\...\\...\\...\\...\\...
%-------------------------------------------------------------------------
%\nocite{ex1,ex2}
\bibliographystyle{apalike}
\bibliography{archivoRef}



\end{document}
