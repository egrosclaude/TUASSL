%--------------------------------------------------------------------
%			1 ASPECTOS ETICOS Y SOCIALES
%--------------------------------------------------------------------

\nota {
	Introducción. Software Libre y Código Abierto. Aspectos éticos. Implicancias sociales. Proyectos libres. Localización.
}
\section{Software Libre y Código Abierto}



\nota {Historia anterior}
\nota {Ética - Valores - Licencias - Libertades - Código Fuente -
Contenidos libres - Programas ejecutables - Distribuciones -
Estándares abiertos - Desarrolladores - Usuarios - Soporte - Organizaciones }

\begin{itemize}
	\item Software Libre, Open Source/Código Abierto, FOSS o FLOSS
	\subitem \url{http://drupal.usla.org.ar/page/%C2%BFque-es-el-software-libre}
	\subitem Free $\neq$ Gratis
	\item El proyecto GNU y la FSF
	\subitem \url{http://www.fsfla.org}
	\subitem \url {http://es.wikipedia.org/wiki/Portal:Software_libre}
	\item Las cuatro libertades
\begin{enumerate} \setcounter{enumi}{-1}
	\item De correr el software, con cualquier propósito
	\item De estudiar cómo está hecho, para poder adaptarlo a sus propias necesidades
	\item De copiarlo y darlo a otras personas para poder ayudarlas
	\item De mejorarlo y donar el resultado a la comunidad, para permitir el avance colectivo
\end{enumerate}
	\item Código fuente, código objeto
	\item Open Source Initiative
	\subitem \url{http://opensource.org/} 
	\item The Open Source Definition - \url{http://www.opensource.org/docs/osd}
	\begin{enumerate}	\item Libre Redistribución
	\item Código Fuente
	\item Obras derivadas
	\item Integridad de los fuentes del autor
	\item No Discriminación contra personas o grupos
	\item No Discriminación contra campos de actividad
	\item Distribución de la licencia
	\item La licencia no debe ser específica de un producto
	\item La licencia no debe restringir el uso de otro software
	\item La licencia debe ser tecnológicamente neutra
\end{enumerate}
\end{itemize}


\figura {mapa}{Mapa conceptual del Software Libre (René Mérou)}{Mapa_conceptual_del_software_libre.png}



\subsection {Aspectos éticos}
\begin{itemize}
	\item Valores
	\subitem Cooperación más importante que individualidades
	\subsubitem \quotes{Si los dos tenemos una manzana/una idea}
	\subsubitem \quotes{Si no está hecho es porque no lo hiciste}
	\subsubitem \quotes{Con suficientes ojos se detectan todos los errores}
	\subitem Puedo aprovechar lo que me ofrecen libremente en
lugar de \emph{piratear}
	\subitem Puedo comprender los mecanismos, trascender la herramienta y acceder a los conceptos
	\subitem Puedo ser parte del desarrollo de la tecnología y cooperativamente contribuir al mejoramiento

	\item Implicancias sociales
	\subitem Equilibrar la balanza de pagos internacional del conocimiento
	\subitem Crítica al concepto de Propiedad Intelectual
		\subsubitem \url{http://www.vialibre.org.ar/}
	\subitem Traducciones y localizaciones
	\subitem Uso correcto de los dineros públicos
	\subitem Transparencia de uso de la información pública
	\subitem Gobierno electrónico
	\subitem Hacktivismo
	\subsubitem Uso de computadoras para promover fines políticos, principalmente libertad de expresión, derechos humanos y ética de la información
	\subsubitem Empoderamiento de las minorías
	\subsubitem Wikileaks
	\subsubitem Diaspora
\end{itemize}


\subsection{Proyectos Libres}

Proyectos que generan conocimiento libre

Cualquier proyecto, sobre cualquier temática, ligado a licencias que permitan el uso, copia, modificación y distribución libre de los conocimientos o la información que allí confluyen. Unen a personas con iguales objetivos o problemáticas, que comparten trabajo y hacen públicos y libres sus resultados.

Libertades
\begin{itemize}
	\item de usar el conocimiento generado en el proyecto para cualquier fin
	\item de estudiar el proyecto y adaptarlo a las propias necesidades
usando la información generada por el proyecto
	\item de redistribuir copias de esa información, de manera que otros se beneficien de ella
	\item de mejorar el proyecto y hacer públicas las mejoras a los demás, de modo que toda la comunidad se beneficie de ellas
	\subitem Open Source Hardware
	\subitem Redes Libres
	\subitem Project Gutenberg
	\subitem Wikipedia, Wikimedia
	\subitem Libros libres, música libre
\end{itemize}



\subsection{Preguntas}
\begin{enumerate}
	\item ¿Cómo se autodefine la Free Software Foundation (FSF)?
	\item ¿En qué consisten las cuatro libertades definidas por FSF?
	\item ¿Puede imaginar una definición diferente de libertades? ¿Puede enunciar otro conjunto de libertades que garanticen los derechos del usuario considerados importantes por FSF?
	\item ¿Cómo se resumen las diferencias entre código fuente y código binario, objeto o ejecutable? ¿Todo código ejecutable es no fuente? ¿Todo archivo fuente es de código?
	\item ¿Por qué es importante la libertad de acceso al código fuente?
	\item ¿Qué significa \emph{privativo} en la terminología de FSF?
	\item ¿De qué manera el software propietario o privativo vulnera las libertades establecidas por FSF?
	\item ¿De qué manera las licencias libres impiden que las empresas que producen software propietario puedan apropiarse del trabajo de quienes desarrollan proyectos libres?
	\item ¿Qué es Copyright? ¿En qué momento aparece el Copyright y a quién pertenece? 
	\item ¿Qué es Copyleft? ¿Cuál es la diferencia entre Copyright y Copyleft?
	\item ¿Qué es \emph{dominio público}? 
	\item El software Windows 8 es descargable de un sitio web\footnote{\url{http://windows.microsoft.com/en-au/windows/download-shop}}. Verdadero o falso:  
	\begin{enumerate} 
		\item Luego, es software libre.
		\item Debe pagarse para poder usarlo, luego no es software libre. 
	\end{enumerate}

	\item ¿Son ejemplos de Software Libre los siguientes?
\begin{itemize}
	\item Internet Explorer, Firefox, Chrome, Opera
	\item Microsoft Office, Libre Office
	\item Adobe Reader, evince, atril
	\item Un ERP como Adempiere o SAP
	\item Tango Gestión, de Axoft
	\item MySQL, Microsoft SQL Server
	\item El web server Apache, el web server IIS
	\item Moodle, Joomla, Plone, Drupal
	\item Android, OSX, ClearOS, Solaris
	\item Aplicaciones que conozca para smartphones 
	\item Un juego web cualquiera que conozca
	\item Extensiones de Chrome para aprender idiomas u otros utilitarios 
	\item Drivers para impresoras
	\item El software web de administración de un router
\end{itemize} 



\end{enumerate}

