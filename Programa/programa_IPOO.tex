\documentclass[11pt]{article}
\usepackage[utf8]{inputenc}
\usepackage{fancyhdr}
\usepackage{graphicx}
\usepackage{longtable}
\usepackage[top=3cm,bottom=2cm,left=2cm,right=2cm]{geometry}

\newcommand{\headerFAI}{\includegraphics[totalheight=2.5cm,scale=1.5]{logodpto.jpg}}

\pagestyle{fancy}
\lhead{\ }
\rhead{\ }
\chead{\headerFAI }
\rfoot{\ }
\lfoot{\ }
%\cfoot{\ }
\renewcommand{\headrulewidth}{0pt}

\begin{document}
\hyphenation{pro-pues-to}
\hyphenation{pro-pues-tos}
\ \\  \ \\ 

\begin{center}
\begin{longtable}{|l|}
  \hline
\textbf{DEPARTAMENTO: } {\sf Programación} \\ \hline 
\\
\textbf{ASIGNATURA: } {\sf \bf \Large Introducción a la Programación Orientada a Objetos} \\ \\ \hline
\parbox[t][0.5in]{5in}{\textbf{CARRERA:} Tecnicatura Superior en Desarrollo de Aplicaciones Web} 
% planes (por ahora) 0085/10 para Web y 0086/10 para BD
\vline \textbf{PLAN:} 0085/10 \\ \hline
\parbox[t][0.5in]{5in}{\textbf{CUATRIMESTRE:} Segundo} 
\vline \textbf{AÑO:} 2012 \\ \hline
% revisar los planes, pero si la materia se dicta un dia la cantidad es 80 hs, si se dicta dos dias es 160 hs.
\textbf{HORAS DE CLASES TOTALES: 160 horas}. Horario semanal:   \\   
 \begin{minipage}[t]{6.5in}                                            
HORAS Y HORARIOS DE TEORÍA: 4 horas- Lunes y Jueves de 16:30hs a 18:30hs \\% 72 horas
HORAS Y HORARIOS DE LABORATORIO: 4 horas- Martes y Jueves de 18:30hs a 20:30hs   \\%72 horas
HORAS DE CONSULTA: 2 hs. \\% 36 hs
HORAS ESTIMADAS EXTRACLASE DE DEDICACIÓN DEL ALUMNO: 4 horas \\% 32 horas 
\end{minipage}
\ \\  \hline
 
\textbf{EQUIPO DE CÁTEDRA:}    \\

\noindent Dr. Luis Reynoso (ASD-2)    \\
Lic. Viviana Sánchez (ASD-3)   \\
Ing. Miriam Lechner (AYP-3) \\
An. Carina Noda (AYP-3)   \\
 \\ \hline
 \ \\
\textbf{OBJETIVOS DE LA MATERIA:}                  \\

\begin{minipage}[t][2.3in]{6.5in}
 \small
\ \\
El objetivo fundamental es la introducción al paradigma de orientación a objetos mediante
la  resolución de problemas de simple y mediana complejidad. Este objetivo fundamental se debe
cumplir incluyendo algunas de las tres siguientes etapas:\\
\begin{itemize}
 \item Adquirir habilidad en la detección de una situación de problema y en el planteo de los posibles
modelos de solución utilizando conceptos básicos de diagramas de clase UML.

 \item Resolver los problemas dados en un lenguaje de programación orientado a objetos.

 \item Resolver problemas en los cuales la aplicación orientada a objetos requiera persistir información.
\end{itemize}
%\\

%\\

Además se pretende que el alumno adquiera conocimientos de los conceptos y terminología básicos
del paradigma orientado a objetos y que adquiera los conocimientos necesarios que serán básicos en asignaturas posteriores.
%\ \\
\end{minipage}
%\                                               \\    \hline                    
\pagebreak
  \hline                    
\textbf{CONTENIDOS MÍNIMOS: }.               \\ 
\begin{minipage}[t]{6.5in}
%\ \\
% segun plan
Introducción a la Programación Orientada a Objetos. Clases y Métodos. Objetos. Encapsulamiento. Polimorfismo. Herencia.
Manejo de datos accedidos a través de un gestor de Base de Datos Relacional.
\vfill
\end{minipage}\                                               \\    \hline                    

   \hline                    
 \ \\
\textbf{PROGRAMA ANALÍTICO:}       \\            \\  

       
\begin{minipage}[t]{6.5in}
%\paragraph{Unidad 1:}

\textbf{Unidad 1- Repaso de Introducción a la Programación.} \\
Tipos de Datos Primitivos. Expresiones. Flujo de Control: secuenciación, sentencias alternativas, sentencias repetitivas.
Documentación de programación. \\ 

\textbf{Unidad 2- Introducción al Paradigma Orientado a Objetos.} \\
Definiciones básicas. Abstracción de Datos. Clases y Métodos. Ocultamiento de la información y encapsulamiento. 
Diagrama de clases UML. Objetos y Referencias. El operador punto (.). Declaración y Creación de objetos.
 Métodos modificadores y observadores. Reglas de Visibilidad: Métodos privados y públicos. Constructores.  \\
 
\textbf{Unidad 3- Herencia.} \\
Colaboración entre clases. Notación en UML. Colaboración con un objeto y con colecciones de objetos. Operaciones asociadas con la
colaboración. Implementación con ArrayList. \\

\textbf{Unidad 4- Herencia.} \\
Introducción a la Herencia. Sobrecarga. Regla de visibilidad protected. El constructor y super. Metodos y clases finales.
Sobreescritura de métodos. Métodos y clases abstractos. Polimorfismo.  \\ 

\textbf{Unidad 5- Persistencia de la Información.} \\
Conceptos básicos. Persistencia. Manejo de memoria primaria y secundaria. Interacción con base de datos relacionales. Sentencias y métodos principales.  \\ 


\ \\
\end{minipage}
                 
\                                               \\    \hline                    
\textbf{PROPUESTA METODOLÓGICA:}                 \\ 
 \begin{minipage}[t]{6.5in}  
Dictado de clases teórico-prácticas y de laboratorio. La materia incluye la aprobacion de dos parciales (con sus recuperatorios
a final de cuatrimestre) y un trabajo final que integrará los contenidos desarrollados en el curso.
Tanto en las clases prácticas como en las de laboratorio se propone que cada docente
organice un grupo de alumnos donde se inicien prácticas de aprendizaje y trabajo colaborativos.
\end{minipage}                         
\                                               \\   
 \hline    

%///
\pagebreak
  \hline    

                \ \\
\textbf{CONDICIONES DE ACREDITACIÓN Y EVALUACIÓN:}  \\ 
\ \\
 \begin{minipage}[t]{6.5in}  

El cursado de la asignatura se acredita con 2 parciales aprobados (o sus respectivos recuperatorios) y el trabajo final.
La condición de aprobado se obtiene con examen final.
\end{minipage} 
\ \\                
\                                               \\    \hline      
              
\textbf{HORARIOS DE CONSULTA DE ALUMNOS:}                 \\  \\
 \begin{minipage}[t]{6.5in}                         
\noindent 
Luis Reynoso: Miércoles de 16:00 a 17:00.- \\
Viviana Sanchez: Viernes 15:00 a 16:00.-\\ 
Carina Noda: Martes 16:00 a 17:00.-   \\
Miriam Lechner: Martes 17:00 a 18:00.-\\ 

Consultas extras antes de los parciales acordadas con los alumnos. \\
Consultas on-line vía PEDCO, acordada con los alumnos.\\
Consultas vía e-mail en forma permanente.\\

\end{minipage}
\                                               \\    \hline    
%///
\ \\                
\textbf{BIBLIOGRAFÍA BÁSICA:}                       \\   
\begin{minipage}[t]{7in}
\begin{itemize}
\item Apuntes de cátedra.
\item Savitch, Walter, Carrano, Frank. {\it Java: An Introduction to Problem Solving \& Programming}. 5th Ed. Editorial: Pearson Prentice Hall. 2008.
\item Pólya, George. {\it How to solve it}. Editorial: Princeton University Press. Princeton, New Jersey. 1973.
\end{itemize}

\noindent Soporte en Internet: http://pedco.uncoma.edu.ar (Facultad de Informática – Programación – Introducción a la Programación Orientada a Objetos)
\end{minipage}

\ \\
\                                               \\    \hline                    
    
\textbf{BIBLIOGRAFÍA DE CONSULTA:}                       \\    
\begin{minipage}[t]{7in}         
\begin{itemize}
\item Grogono, P., Nelson, S. {\it Problem Solving and Computer Programming}. Addison Wesley 1982.
\item Fontanela, Carlos. {\it Orientación a Objetos con Java y UML}. Editorial Nueva Librería 2004.
\end{itemize}          
\end{minipage}
\                                               \\    

\begin{minipage}[t]{7in}         
\ \\
\end{minipage}
\                                               \\    \hline                    
\parbox[b][1.0in]{3.5in}{\center \small \textbf{FIRMA DEL PROFESOR}}
\vline 
\parbox[b][1.0in]{3.5in}{\center \small \textbf{FIRMA DEL JEFE DE DEPARTAMENTO}}       \\   \hline
%\end{tabular}
\end{longtable}
\end{center}


\end{document}
