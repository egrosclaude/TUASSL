
\section{Objetivos}
\subsection{De la carrera}
Según el documento fundamental de la Tecnicatura, el Técnico Superior en Administración de Sistemas y Software Libre estará capacitado para:
\begin{itemize}
	\item Desarrollar actividades de administración de infraestructura. Comprendiendo la administración de sistemas, redes y los distintos componentes que forman la
infraestructura de tecnología de una institución, ya sea pública o privada.
	\item Aportar criterios básicos para la toma de decisiones relativas a la adopción de nuevas tecnologías libres.
	\item Desempeñarse como soporte técnico, solucionando problemas afines por medio de la comunicación con comunidades de Software Libre, empresas y desarrolladores de
software.
	\item Realizar tareas de trabajo en modo colaborativo, intrínseco al uso de tecnologías libres.
	\item Comprender y adoptar el estado del arte local, nacional y regional en lo referente a implementación de tecnologías libres. Tanto en los aspectos técnicos como legales.
\end{itemize}
\subsection{De la asignatura}
\begin{itemize}
\item Conocer los aspectos técnicos, legales, económicos y sociales que distinguen al Software Libre y de Código Abierto
\item Conocer las formas de analizar, evaluar y utilizar las fuentes de documentación y soporte del Software Libre y de Código Abierto
\end{itemize}


\section{Cursado}
\begin{itemize}
	\item Cuatrimestral de 16 semanas, 64 horas totales
	\item Clases teórico-prácticas presenciales
	\item Promocionable con trabajos prácticos
\end{itemize}


\section {Contenidos}
\subsection{Contenidos mínimos}
\begin{itemize}
\item Las licencias de software. Software Libre y Open Source. Comparación. 
\item Ventajas de la disponibilidad del código fuente. 
\item Modelos de desarrollo abiertos y colaborativos. 
\item Aspectos legales y de explotación del Software Libre. 
\item Implantación de sistemas de Software Libre. Factibilidad. 
\item Aspectos económicos y modelos de negocio del Software Libre. 
\item Costo total de operación. Comparación con otras alternativas. 
\item El Software Libre en el sector público, en la educación y en la empresa.
\end{itemize}

\subsection {Programa}
\begin{enumerate}
	\item Introducción. Software Libre y Código Abierto. Aspectos éticos. Implicancias sociales. Localización. Proyectos libres.
	\item Aspectos técnicos. Proyectos de Software Libre. Modelo de desarrollo. Infraestructura tecnológica. Manejo de documentación y soporte. Seguridad. 
	\item Aspectos legales. Dominio Público, Copyright, Copyleft y Licenciamiento. Licencias de FSF, Creative Commons, OSI, otras. 
	\item Uso y aplicación de Software Libre. Costo total de operación. Estudio de costo/beneficio. Procesos de migración. 
	\item Producción de Software Libre y con Software Libre. Modelos de negocio. Colaboración en proyectos. Organizaciones y software.  Administración pública, educación pública, sector privado.
\end{enumerate}

\subsection {Bibliografía inicial}
\begin{itemize}
	\item \textbf{Introducción al Software Libre}, Jesús González Barahona, Joaquín Seoane Pascual y Gregorio Robles
	\item \textbf{Aspectos legales y de explotación del software libre}, Malcom Bain, Manuel Gallego Rodríguez, Manuel Martínez Ribas y Judit Rius Sanjuán
	\item \textbf{Guía práctica sobre Software Libre, su selección y aplicación local en América Latina y el Caribe}, Fernando Da Rosa y Federico Heinz
\end{itemize}



\section{Evaluación}
La evaluación de la materia se realizará mediante trabajos grupales de investigación y desarrollo sobre proyectos de Software Libre, de la siguiente manera.
\begin{itemize}
	\item Los estudiantes se dividirán en grupos de 2 a 5 personas. 
	\item Los grupos desarrollarán trabajos prácticos en etapas que se distribuirán a lo largo de la materia. 
	\item Cada grupo abrirá un diario, blog o wiki de acceso público en cualquier sitio disponible y publicará, mediante el Foro de la materia, la forma de acceder al diario para lectura. Los docentes y los demás estudiantes de la materia podrán acceder al diario del grupo para lectura. Todo cambio en la dirección o forma de acceso deberá ser informado mediante el Foro.
	\item El grupo irá aportando los resultados de cada etapa de los trabajos a su diario, y periódicamente comentará además en clase las experiencias surgidas durante la realización de los trabajos.
	\item El material publicado en el diario será reunido en un documento final que será entregado \textbf{en formato electrónico} al finalizar la materia. El documento indicará tema del trabajo, resumen, integrantes del grupo, desarrollo y conclusiones. 
	\item El documento será acompañado por una presentación de no más de treinta minutos que será expuesta según el cronograma adjunto. 
	\item La acreditación final tendrá en cuenta la calidad del material aportado al diario por el grupo, la calidad de los documentos finales de los trabajos, la presentación oral y la participación en clase ofreciendo la experiencia adquirida durante la realización de los trabajos.
\end{itemize}

\subsection {Trabajo I - Colaboración con proyectos libres}
\subsubsection{Etapa 1}  
Descargar e instalar software ofrecido por un proyecto de Software Libre que esté en actividad (puede tratarse de un entorno de escritorio, un programa de sistema, programas de usuario final, una distribución completa, etc.). Familiarizarse con el software utilizándolo. 
\subsubsection{Etapa 2} 
Basándose en el conocimiento adquirido con el uso del software, colaborar de alguna forma con el proyecto que lo origina: 
\begin{itemize}
	\item traduciendo o localizando parte del software,
	\item generando documentación faltante, 
	\item traduciendo parte de la documentación, 
	\item detectando y denunciando errores en el software o en la documentación,
	\item aportando, modificando o corrigiendo código,
	\item aportando conocimiento a los usuarios del proyecto en blogs, salas de chat, bases de conocimiento, etc.
\end{itemize}
Puede abordarse cualquier cantidad manejable de proyectos. La colaboración debe consistir en alguna interacción positiva y completa con cada proyecto. El grupo incorporará al diario los reportes que acrediten esa interacción. Cuando no sea posible realizar o completar la interacción se indicarán las causas, y las acciones realizadas.

El aporte al proyecto debe efectuarse por los canales establecidos por el proyecto. Si se trata de documentación, respetar el formato utilizado; si es el reporte de un error, hacerlo por la vía preferida por el proyecto, etc.

\subsubsection{Etapa 3} 
El grupo entregará un documento conteniendo la historia de las interacciones con cada proyecto, adjuntando las pruebas en anexos y ofrecerá una presentación.

\subsection {Trabajo II - Evaluación de proyectos libres}

\subsubsection{Etapa 1} 
El grupo enunciará un determinado requerimiento concreto de software que puede ser presentado por un empleador. Algunos ejemplos posibles son:
\begin{itemize}
	\item \quotes{un servidor de correo electrónico que maneje listas},
	\item  \quotes{una aplicación de control de asistencia para empleados},
	\item  \quotes{un sistema de edición de textos para traductores},
	\item  \quotes{un sistema de gestión de contenidos web que incluya workflow}, 
	\item \quotes{un motor de juegos 2D para crear juegos que asistan en la enseñanza de matemática},
	\item  \quotes{un programa de simulación de ataques para evaluar postura de seguridad}, 
	\item \quotes{un sistema de control de stock para zapaterías},
	\item \quotes{una distribución de GNU/Linux para escuelas de arte},
	\item \quotes{una distribución para sistemas empotrados}, etc.
\end{itemize}
El grupo debe comprender el propósito del software requerido y debe contar con al menos un integrante con conocimiento razonable de la temática involucrada. El grupo escribirá una entrada en el diario consignando toda la información posible sobre los requerimientos. 

\subsubsection{Etapa 2} 

\begin{itemize}
	\item El grupo $n$ (en adelante \quotes{el proveedor}) tomará a su cargo el requerimiento del grupo $n+1$ (en adelante \quotes{el cliente}), y se atendrá a dicha descripción para el resto del trabajo. 
	\item El grupo proveedor buscará proyectos de SL que apunten a cubrir esos requerimientos y seleccionará al menos dos proyectos, idealmente tres, de entre ellos.
\end{itemize}

\subsubsection{Etapa 3}
Los proyectos serán comparados en función de varios parámetros o dimensiones.
\begin{itemize}
	\item  ajuste a los requerimientos (actual, previsto o potencial),
	\item  licenciamiento, 
	\item  motivación del desarrollo, 
	\item  modelos de negocio del proyecto, 
	\item  tamaño y permanencia de la comunidad,
	\item  dinámica de soporte, 
	\item  dinámica de actualizaciones y mejoras del software.
\end{itemize}

Se podrán agregar a la comparación uno o más desarrollos no libres. 

Las dudas sobre detalles de los requerimientos serán dirigidas al grupo cliente, y contestadas por aquél, mediante el Foro de la página de la materia.  
\subsubsection{Etapa 4} 
El grupo entregará un documento conteniendo la comparación y haciendo una recomendación final, explicando sus fundamentos. Deberán volcar en el trabajo lo que se vaya aprendiendo durante el curso de la materia, en cada uno de los parámetros o dimensiones nombrados. Finalmente ofrecerán una presentación sobre el trabajo.

\label{sub:acreditacion}

\subsection {Cronograma de ejecución}
\begin{tabular}{c|l|l|l}
Semana & Unidad & Trabajo I & Trabajo II\\
\hline
\hline
1	& 	1. Introducción, Software Libre & Etapa 1 &  \\
2 	& 								 	& \\
\hline
\hline
3	& 	2. Aspectos técnicos			& Etapa 2 &  \\
4 	& 									&\\
5	& 									&\\
6	& 									&\\
\hline
\hline
7 	& 	3. Aspectos legales				& Etapa 3 \\
8	& 									& Entrega y presentaciones\\ 
9	& 									& & Etapas 1 y 2\\
\hline
\hline
10	& 	4. Uso de SL					&& Etapa 3\\ 
11	& 									& \\
12	& 									&\\
13	& 									&\\
\hline
\hline
14	& 	5. Producción de SL				&& Etapa 4\\
15	& 									&\\
16	& 									&& Entrega y presentaciones\\ 
\hline
\end{tabular}



% \begin{tabular}{|r|c|c|c|c|c|c|c|c|}
% \hline
%\textsf{7} & fbox {algo} & & & & & & &\\ 
%\hline
%\textsf{7} & & & & & & & &\\ 
%\hline
%\end{tabular}

% subsection  (end)
